\documentclass[a4paper,doc2]{ltxdoc} % doc2 is needed to force the old version, or links get colored in a weird red way even with hidelinks. https://github.com/latex3/latex2e/issues/822

%%%%%%%%%%%%%%%%%%%%%%%%%%%%%%
%%% Packages
%%%%%%%%%%%%%%%%%%%%%%%%%%%%%%
%% Warning: if you compile and get:
%% ERROR: Argument of \tikz@lib@matrix@with@options has an extra }.
%% make sure to fix catcodes around it as | is given a different meaning in ltxdoc.

\usepackage{amsmath}
\usepackage[margin=3cm]{geometry}
\usepackage{calc}
\usepackage{tikz}
\usetikzlibrary{shadows,fit}
% \usetikzlibrary fails because file is not in current directory, lazy to setup TEXINPUTS
\makeatletter
  \ProvidesPackage{robust-externalize}[1.0+unstable Cache anything (tikz, latex, python) in a robust, efficient and pure way.]
% todo: understand why python scripts with raise NameError("42") do not make latex crash.

\RequirePackage{pgfkeys} % We use the /robExt/... path to store our keys.
\RequirePackage{pgffor} % For the .list keys
\RequirePackage{graphicx} % For the includegraphics command
\RequirePackage{verbatim} % For the \verbatim command, useful for debugging purpose for instance
\RequirePackage{xsimverb} % To easily write verbatim code to files
\RequirePackage{etoolbox} % To easily write verbatim code to files

%% TODO list:
% - provide an easy way to use cross-ref, bibtex etc (we just need to add them when writing the file) without recompiling the whole document (we don't want to lose the cache everytime a new bib entry is added) but while preserving.
% - create pre-made settings for tikz, tikz-cd, ...
% - check compatibility with windows
% - write documentation

%%% Under the hood, this library is quite simple: each picture must, somehow, provide:
%% - \l_robExt_final_file LaTeX3 string containing the content of the final file
%% - 
%% Then, the library will hash everything to create a unique name (of the content, the template code, and the set of dependency filenames),
%% it will create a file "MD5.tex" containing the pre-template+content+post-template, and it will compile it.

%%%%%%%%%%%%%%%%%%%%%%%%%%%%%%
%%%%%%%%%%% Utils %%%%%%%%%%%%
%%%%%%%%%%%%%%%%%%%%%%%%%%%%%% 

%%% Utils:
% https://tex.stackexchange.com/questions/690700/latex3-elegant-way-to-forward-a-variable-outside-of-the-group
% modified to deal with csname instead
\def\robExtKeepaftergroup#1{%
   %\global \expandafter \expandafter \let \csname x:#1\endcsname =\csname #1\endcsname
   \global \expanded{\noexpand \let \expandafter\noexpand\csname x:#1\endcsname =\expandafter\noexpand\csname #1\endcsname}
   \aftergroup\let
   \expandafter\aftergroup\csname #1\endcsname%
   \expandafter\aftergroup \csname x:\string#1\endcsname
}

\ExplSyntaxOn
%% See also https://tex.stackexchange.com/questions/695432/latex3-latex-doubles-the-number-of-hashes-when-storing-them-in-string/695460#695460
\cs_generate_variant:Nn \str_replace_all:Nnn { Nnx }
\cs_generate_variant:Nn \str_replace_all:Nnn { cnx }
\cs_generate_variant:Nn \str_replace_all:Nnn { cxx }
\cs_set:Nn \str_set_hash_robust:Nn {
  \str_set:Nn {#1} {#2}
  \str_replace_all:Nnx {#1} { ## } { \c_hash_str }
}
\cs_generate_variant:Nn \str_set_hash_robust:Nn { cn }
\cs_generate_variant:Nn \str_set_hash_robust:Nn { cx }

% Double the number of hashes... quite dirty but cannot find any solution or the user need to double it itself
\NewDocumentCommand{\robExtStrSetDoubleHash}{mm}{
  \str_set:Nn {#1} {#2}
  \str_replace_all:Nnx {#1} { ## } { \c_hash_str \c_hash_str }
}

\ExplSyntaxOff

%%%%%%%%%%%%%%%%%%%%%%%%%%%%%%
%%%%%%%%%% SCRIPTS %%%%%%%%%%%
%%%%%%%%%%%%%%%%%%%%%%%%%%%%%% 

%%% Create scripts to remove useless files:
%%% Note that we don't override the script if it exists on purpose (the user might have changed it to fits his needs)
\begin{filecontents}[noheader]{robExt-remove-old-figures.py}
#!/usr/bin/env python3
import os
import re
import glob
# Just run this script in order to remove all old figures not listed in robExt-all-figures.txt.

# Note that this part is not extracted from the pdf file since it might be different on a previous run. You can however hardcode
# it here, your updated script will not be overriden unless you remove it yourself.
prefixes = [ "robExt-" ]
folders  = [ "robustExternalize" ]

def main():
    imagesToKeep = dict()
    list_all_figures_file = glob.glob('*robExt-all-figures.txt')
    for filename in list_all_figures_file:
        with open(filename) as f:
            for line in f:
                line = line.strip()
                if line.endswith('.tex'):
                    imagesToKeep[line[:-4]] = True # The exact value is not important, we mostly use dict to get ~O(1) access

    listOfFilesToRemove = []
    # We are looking for images in the folders
    for folder in folders:
        for root, dirs, files in os.walk(folder):
            for f in files:
                for prefix in prefixes: # Not the most efficient, but anyway we typically have a single prefix
                    # In case prefix contains weird caracters that collide with regexps:
                    prefixEsc = re.escape(prefix)
                    # result_search = re.search(rf"^({prefixEsc}[A-F0-1]{32}).*", f)
                    result_search = re.search(rf"^(.*[A-F0-9]{{32}}).*", f)
                    if result_search:
                        if result_search.group(1) not in imagesToKeep:
                            listOfFilesToRemove.append(os.path.join(root,f))
    for f in listOfFilesToRemove:
        print(f"-- {f}")
    print(f"Above are the files to remove, are you sure you want to proceed? [y/N] (based on prefixes {prefixes})")
    x = input().strip()
    if x not in ["y", "Y"]:
        print("All right, we abort.")
        exit(1)
    for f in listOfFilesToRemove:
        os.remove(f)
        print(f"Removed {f}")
        
if __name__ == '__main__':
    main()
\end{filecontents}

\ExplSyntaxOn

%%%%%%%%%%%%%%%%%%%%%%%%%%%%%%%%%%%%%%
%%%%%%%%%%% Compatibility %%%%%%%%%%%%
%%%%%%%%%%%%%%%%%%%%%%%%%%%%%%%%%%%%%%
% These commands exist on really recent kernels (~june 2023), so we provide them in case they do not already exist

\ProvideDocumentCommand \NewEnvironmentCopy {mm} {
    \expandafter \NewCommandCopy \csname#1\expandafter\endcsname \csname#2\endcsname
    \expandafter \NewCommandCopy \csname end#1\expandafter\endcsname \csname end#2\endcsname
}

\ProvideDocumentCommand \RenewEnvironmentCopy {mm} {
    \expandafter \RenewCommandCopy \csname#1\expandafter\endcsname \csname#2\endcsname
    \expandafter \RenewCommandCopy \csname end#1\expandafter\endcsname \csname end#2\endcsname
}

\ProvideDocumentCommand \DeclareEnvironmentCopy {mm} {
    \expandafter \DeclareCommandCopy \csname#1\expandafter\endcsname \csname#2\endcsname
    \expandafter \DeclareCommandCopy \csname end#1\expandafter\endcsname \csname end#2\endcsname
}

\ProvideDocumentCommand \RenewEnvironmentCopy {mm} {
    \expandafter \RenewCommandCopy \csname#1\expandafter\endcsname \csname#2\endcsname
    \expandafter \RenewCommandCopy \csname end#1\expandafter\endcsname \csname end#2\endcsname
}

%%%%%%%%%%%%%%%%%%%%%%%%%%%%%%
%%%%%%%%%%% Paths %%%%%%%%%%%%
%%%%%%%%%%%%%%%%%%%%%%%%%%%%%%

\def\robExtPrefixFilename{robExt-}

\NewExpandableDocumentCommand{\robExtAddCachePathAndName}{m}{%
  \ifdefined\robExtCacheFolder%
    \robExtCacheFolder%
  \fi\robExtPrefixFilename#1%
}

\NewExpandableDocumentCommand{\robExtAddCachePath}{m}{%
  \ifdefined\robExtCacheFolder%
    \robExtCacheFolder%
  \fi#1%
}


\NewDocumentCommand{\robExtCheckIfPrefixFolderExists}{}{
  % Check if the output directory exists
  \ifdefined\robExtCacheFolder
    %% Not sure how to do an "OR" otherwise

    \bool_if:nTF { \sys_if_shell_unrestricted_p: || \cs_if_exist_p:N {\robExtForceCompilation}}
    {
      \ifdefined\robExtDoNotMkdirFolder\else
        \ifdefined\robExtManualMode
          \message{If ~ you ~ get~ an~ error,~ make ~ sure ~ to ~ enable ~ pdflatex ~ -shell-escape ~ or ~ to ~ manually ~ create ~ the ~ folder ~ \robExtCacheFolder.}
        \else
          \sys_shell_now:x {mkdir ~ -p ~ \robExtCacheFolder}
        \fi
      \fi
    }{
      \message{If ~ you ~ get~ an~ error,~ make ~ sure ~ to ~ enable ~ pdflatex ~ -shell-escape ~ or ~ to ~ manually ~ create ~ the ~ folder ~ \robExtCacheFolder.}
    }
  \fi
}

\NewExpandableDocumentCommand{\robExtGetPrefixPath}{}{%
  \ifdefined\robExtCacheFolder%
    \robExtCacheFolder%
  \fi%
}


\NewExpandableDocumentCommand{\robExtAddPrefixName}{m}{%
  \robExtPrefixFilename#1%
}

%% Todo: not sure if I should use \seq_push:Nx \l_file_search_path_seq {subfolder}
%% to find the subfolder (seems to work for input/includegraphics/...), or if it's
%% better to hardcode the subfolder.

%%%%%%%%%%%%%%%%%%%%%%%%%%%%%%%%%%%%%%%%%%%%%%%%%%%%%%%%%
%%%%%%%%%%% Setup new commands and variables %%%%%%%%%%%%
%%%%%%%%%%%%%%%%%%%%%%%%%%%%%%%%%%%%%%%%%%%%%%%%%%%%%%%%%

\cs_generate_variant:Nn\seq_remove_all:Nn { NV }
\cs_generate_variant:Nn\tl_rescan:nn { nv }
\cs_generate_variant:Nn\tl_set_rescan:Nnn { Nnv }
%\cs_generate_variant:Nn\tl_set_rescan:cnn { cnv }
\cs_generate_variant:Nn \iow_now:Nn { NV }
\cs_generate_variant:Nn \iow_now:Nn { Nx }
\cs_generate_variant:Nn \iow_open:Nn { Nx }
\cs_generate_variant:Nn \ior_str_get:NN { Nc }
\cs_generate_variant:Nn \str_replace_all:Nnn { NnV }
\cs_generate_variant:Nn \str_replace_all:Nnn { Nnx }
\cs_generate_variant:Nn \str_replace_all:Nnn { Nnv }
\cs_generate_variant:Nn \file_if_exist:nTF { xTF }
\cs_generate_variant:Nn \str_set:Nn { NV }

%% Temporary: used when quickly writing to a file
\iow_new:N \g_robExt_write_iow
\ior_new:N \g_robExt_read_ior
%% This is used to write the full list of figures in a single file (used for instance by Makefile etc...)
%% TODO: create a makefile.
\iow_new:N \g_robExt_write_list_all_figures
%% Create a file robExt-all-figures.txt with the list of .tex files
\iow_open:Nx \g_robExt_write_list_all_figures {\jobname-\robExtAddPrefixName{all-figures.txt}}
\iow_new:N \g_robExt_write_manually_compile_all_missing_figures
\iow_open:Nx \g_robExt_write_manually_compile_all_missing_figures {\jobname-\robExtAddPrefixName{compile-missing-figures.sh}}
% Contains the template:
\str_new:N \l_robExt_template
\str_new:N \l_robExt_final_file

% Contains the list of dependency files (useful to compute the final md5sum)
\seq_new:N \l_robExt_dependencies
% Contains a string where on each line we have: "md5sum, dependency". The first line has nothing as "dependency" as it is the main fine whose final name is the md5sum of the dependencies.
\str_new:N \l_robExt_dependencies_mdfive
% Contains the current compilation command (including the name of the file to compile).
\str_new:N \l_robExt_currentCompilationCommand


%%%%%%%%%%%%%%%%%%%%%%%%%%%%%%%%%%%%%
%%%%%%%%%%% Placeholders %%%%%%%%%%%%
%%%%%%%%%%%%%%%%%%%%%%%%%%%%%%%%%%%%%

%% Placeholders are strings like "__MYLIBRARY__" or "__MAINCONTENT__" that will be replaced by a given content.
%% In practice:
%% - for every placeholder MYPLACEHOLDER, a macro \l_robExt_placeholder_MYPLACEHOLDER_str is created, containing
%%   the string to use to replace it
%% - a sequence \l_robExt_placeholders_seq that is a list of string, contains the list of all placeholders,
%%   in a string, like [MYLIBRARY, MAINCONTENT] etc...
%% One issue is that LaTeX does not keep some symbols (e.g. % etc...) when used inside a macro, so we define
%% different commands depending on whether they can be used in a macro or not, whether they should be taken
%% from an external file etc...
\seq_clear_new:N \l_robExt_placeholders_seq 

\NewDocumentCommand{\robExtShowPlaceholder}{sm}{
  \message{Placeholder ~ #2 ~ contains:^^J~ \use:c{l_robExt_placeholder_#2_str}}
  \IfBooleanTF{#1}{\cs_show:c { l_robExt_placeholder_#2_str }}{}
}

\NewDocumentCommand{\robExtShowPlaceholders}{s}{
  \message{List ~ of ~ placeholders:}
  \seq_map_inline:Nn \l_robExt_placeholders_seq {\message{##1,}}
  \IfBooleanTF{#1}{\cs_show:N \l_robExt_placeholders_seq}{}
}

\NewDocumentCommand{\robExtShowPlaceholdersContents}{s}{
  \message{List ~ of ~ placeholders:}
  \seq_map_inline:Nn \l_robExt_placeholders_seq {\robExtDebugPlaceholder{##1}}
  \IfBooleanTF{#1}{\cs_show:N \l_robExt_placeholders_seq}{}
}

\NewDocumentCommand{\robExtPrintPlaceholderNoReplacement}{sm}{%
  % For some reasons, newlines are displayed as \Omega. We need to replace them with \\
  % https://tex.stackexchange.com/questions/694716/print-latex3-string-verbatim/694717
  \tl_set_eq:Nc \l_robExt_tmp_str { l_robExt_placeholder_#2_str }
  \tl_replace_all:Nnn \l_robExt_tmp_str {^^J} { \par }
  \tl_replace_all:Nnn \l_robExt_tmp_str { ~ } { \  }
  \IfBooleanTF{#1}{\texttt{\use:c{l_robExt_placeholder_#2_str}}}{\begin{flushleft}\ttfamily%
      \l_robExt_tmp_str
    \end{flushleft}%
  }
}
\let\printPlaceholderNoReplacement\robExtPrintPlaceholderNoReplacement

\NewDocumentCommand{\robExtPrintPlaceholder}{sm}{%
  \robExtGetPlaceholderInResult{#2}
  % For some reasons, newlines are displayed as \Omega. We need to replace them with \\
  % https://tex.stackexchange.com/questions/694716/print-latex3-string-verbatim/694717
  \tl_set_eq:NN \l_robExt_tmp_str \l_robExt_result_str
  \tl_replace_all:Nnn \l_robExt_tmp_str {^^J} { \par }
  \tl_replace_all:Nnn \l_robExt_tmp_str { ~ } { \  }
  \IfBooleanTF{#1}{\texttt{\l_robExt_tmp_str}}{\begin{flushleft}\ttfamily%
      \l_robExt_tmp_str
    \end{flushleft}%
  }
}
\let\printPlaceholder\robExtPrintPlaceholder


\NewDocumentCommand{\robExtPrintAllPlaceholdersExceptDefaults}{s}{
  List ~ of ~ placeholders:\\
  \seq_map_inline:Nn \l_robExt_placeholders_seq {
    % We hide the elements starting with __ROBEXT_
    \str_if_in:nnTF { ##1 } { __ROBEXT_ } {
      \IfBooleanTF {#1} {
        - ~ Placeholder ~ called ~ \texttt{\tl_to_str:n {##1}} ~ defined ~ by ~ default ~ (we ~ hide ~ the ~ definition ~ to ~ save ~ space)\\
      }{}
    }{
      - ~ Placeholder ~ called ~ \texttt{\tl_to_str:n {##1}} ~ contains: \robExtPrintPlaceholderNoReplacement{##1}
    }
  }
}
\let\printAllPlaceholdersExceptDefaults\robExtPrintAllPlaceholdersExceptDefaults

\NewDocumentCommand{\robExtPrintAllPlaceholders}{}{
  List ~ of ~ placeholders:\\
  \seq_map_inline:Nn \l_robExt_placeholders_seq {- ~ Placeholder ~ called ~ \texttt{\tl_to_str:n {##1}} ~ contains: \robExtPrintPlaceholderNoReplacement{##1}}
}
\let\printAllPlaceholders\robExtPrintAllPlaceholders


\NewDocumentCommand{\robExtEvalPlaceholderNoReplacement}{m}{
  % \scantokens{\use:c{l_robExt_placeholder_#1_str}}
  % scantokens add an empty space at the end, so we need to add \empty to avoid it having effects
  % https://tex.stackexchange.com/questions/213659/could-someone-further-elucidate-expansion-catcodes-and-scantokens
  %\exp_args:Nx \scantokens {\use:c{l_robExt_placeholder_#1_str}}
  \tl_rescan:nv {}{ l_robExt_placeholder_#1_str }
  % \tl_rescan:nc { } { l_robExt_placeholder_#1_str }
}
\let\evalPlaceholderNoReplacement\robExtEvalPlaceholderNoReplacement

\NewDocumentCommand{\robExtGetPlaceholderNoReplacement}{m}{
  \str_use:c { l_robExt_placeholder_#1_str }
}
\let\getPlaceholderNoReplacement\robExtGetPlaceholderNoReplacement


% Make sure that the placeholder is in the list \l_robExt_placeholders_seq.
% This should automatically be called by other tools
\NewDocumentCommand{\robExtAddPlaceholderToList}{m}{
  % Make sure we have a string here:
  \str_set_hash_robust:Nn \l_robExt_tmp_str { #1 }
  % First we remove existing occurrences (useful to avoid listing the same macro more than once
  % if we redefine a macro):
  \seq_remove_all:NV \l_robExt_placeholders_seq \l_robExt_tmp_str
  \seq_put_left:NV \l_robExt_placeholders_seq \l_robExt_tmp_str
}

\NewDocumentCommand{\robExtRemovePlaceholder}{m}{
  % This seems to be required to ensure catcodes are correct before removing the elements:
  \str_set_hash_robust:Nn \l_robExt_tmp_str { #1 }
  \seq_remove_all:NV \l_robExt_placeholders_seq \l_robExt_tmp_str
  \expandafter \let \csname l_robExt_placeholder_#1_str \endcsname \undefined 
}
\let\removePlaceholder\robExtRemovePlaceholder

%% Usage: \placeholderFromContent{MYTITLE}{My slide title}
%% MYTITLE will contain at the end "My slide title"
%% Warning: only LaTeX-friendly code should be placed here, as LaTeX does not preserve some symbols and adds spaces
%% Tested!
\NewDocumentCommand{\robExtPlaceholderFromContent}{mm}{
  \str_set_hash_robust:cn { l_robExt_placeholder_#1_str } {#2}
  \robExtAddPlaceholderToList{#1}
}
\let\placeholderFromContent\robExtPlaceholderFromContent
\let\robExtSetPlaceholder\robExtPlaceholderFromContent
\let\setPlaceholder\robExtPlaceholderFromContent

\NewDocumentCommand{\robExtPlaceholderFromString}{mm}{
  \str_set_eq:cN { l_robExt_placeholder_#1_str } {#2}
  \robExtAddPlaceholderToList{#1}
}
\let\placeholderFromString\robExtPlaceholderFromString
\let\setPlaceholderFromString\robExtPlaceholderFromString

\NewDocumentCommand{\robExtCopyPlaceholder}{mm}{
  \str_set_eq:cc { l_robExt_placeholder_#1_str } { l_robExt_placeholder_#2_str }
  \robExtAddPlaceholderToList{#1}
}
\let\copyPlaceholder\robExtCopyPlaceholder


%% Add something to the right of the placeholder
%% By default it adds a space in between, unless we use the star version
\NewDocumentCommand{\robExtAddToPlaceholder}{smm}{
  \str_if_exist:cTF { l_robExt_placeholder_#2_str } {
    \str_set_hash_robust:Nn \l_tmp_str {#3} %% needed as put_right does not convert to string first
    \IfBooleanTF{#1}{}{\str_put_right:cn { l_robExt_placeholder_#2_str } { ~ } }
    \str_put_right:cV { l_robExt_placeholder_#2_str } \l_tmp_str
  }{
    \str_set_hash_robust:cn { l_robExt_placeholder_#2_str } {#3}
    \robExtAddPlaceholderToList{#2}
  }
}
\let\addToPlaceholder\robExtAddToPlaceholder

%% Add something to the right of the placeholder
%% By default it adds a space in between, unless we use the star version
\NewDocumentCommand{\robExtAddBeforePlaceholder}{smm}{
  \str_if_exist:cTF { l_robExt_placeholder_#2_str } {
    \str_set_hash_robust:Nn \l_tmp_str {#3} %% needed as put_right does not convert to string first
    \IfBooleanTF{#1}{}{\str_put_left:cn { l_robExt_placeholder_#2_str } { ~ } }
    \str_put_left:cV { l_robExt_placeholder_#2_str } \l_tmp_str
  }{
    \str_set_hash_robust:cn { l_robExt_placeholder_#2_str } {#3}
    \robExtAddPlaceholderToList{#2}
  }
}
\let\addBeforePlaceholder\robExtAddBeforePlaceholder


% Usage:
% \begin{placeholderFromCode}{HELPERFUNCTION}
%   def my_helper_function(bla):
%   return bla + 1
% \end{placeholderFromCode}
% HELPERFUNCTION will contain at the end "def ..."
% This environment cannot be placed inside any other macro/align/...
\NewDocumentEnvironment{RobExtPlaceholderFromCode}{m}{%
  % % debug part
  % \str_set:Nn \l_test_str {#1}
  % \show\l_test_str
  %% Environments can't use verbatim yet: https://github.com/latex3/latex3/issues/591
  % Might be related: https://tex.stackexchange.com/questions/633523/saving-the-body-of-an-environment-to-a-file-verbatim-using-xparse
  %% Here is the first part:
  %% https://tex.stackexchange.com/a/680259/116348 might work and be more efficient, but it might be less reliable
  %% and definitely more complicated and error prone. Instead, we write to a file and read the result.
  %% TODO: try to do it using verbatim, might be trivial or complicated, not sure, maybe see https://tex.stackexchange.com/questions/555359/reading-lines-verbatim-into-a-sequence-variable
  \XSIMfilewritestart{\jobname-robExt-tmp-file-you-can-remove.tmp}%
}{%
  \XSIMfilewritestop%
  \ior_open:Nn \g_robExt_read_ior {\jobname-robExt-tmp-file-you-can-remove.tmp}%
  %% Put the file in l_robExt_tmp_contain_file
  \str_clear_new:N \l_robExt_tmp%
  \ior_str_map_inline:Nn \g_robExt_read_ior {%
    \str_gput_right:Nx \l_robExt_tmp {\tl_to_str:N{##1}^^J}%
  }%
  \str_set_eq:cN {l_robExt_placeholder_#1_str} \l_robExt_tmp%
  \robExtAddPlaceholderToList{#1}%
  %% Otherwise they will be lost when the environment ends
  \robExtKeepaftergroup{l_robExt_placeholders_seq}%
  %% for other variable
  \robExtKeepaftergroup{l_robExt_placeholder_#1_str}%
}%
\let\PlaceholderFromCode\RobExtPlaceholderFromCode
\let\endPlaceholderFromCode\endRobExtPlaceholderFromCode
\let\SetPlaceholderCode\RobExtPlaceholderFromCode
\let\endSetPlaceholderCode\endRobExtPlaceholderFromCode

%% Usage:
%% \placeholderPathFromFilename{MYLIBPATH}{mylib.py}
%% This will copy mylib.py in the cache, and set MYLIBPATH to the name of the file in the cache like
%% MYLIBPATH = robExt-abcmylib.py
%% Tested!
\NewDocumentCommand{\robExtPlaceholderPathFromFilename}{mm}{
  \ior_open:Nn \g_robExt_read_ior {#2}
  %% Put the file in l_robExt_tmp_contain_file
  \str_clear_new:N \l_robExt_tmp_contain_file
  \ior_str_map_inline:Nn \g_robExt_read_ior {
    \str_put_right:Nx \l_robExt_tmp_contain_file {\tl_to_str:N{##1}^^J}
  }
  %% computes the new filename based on the md5
  \str_clear_new:N \l_robExt_tmp_filename
  \str_set:Nx \l_robExt_tmp_filename_no_prefix {\pdfmdfivesum{\l_robExt_tmp_contain_file}#2}
  %% Writes the content to the file
  \robExtCheckIfPrefixFolderExists
  \iow_open:Nx \g_robExt_write_iow {\robExtAddCachePathAndName{\l_robExt_tmp_filename_no_prefix}}
  \iow_now:NV \g_robExt_write_iow \l_robExt_tmp_contain_file
  \iow_close:N \g_robExt_write_iow
  %% sets the template name to the relative path to the file
  \str_set:cx { l_robExt_placeholder_#1_str } {\robExtPrefixFilename\l_robExt_tmp_filename_no_prefix}
  \robExtAddPlaceholderToList{#1}
}
\let\placeholderPathFromFilename\robExtPlaceholderPathFromFilename

%% Usage:
%% \placeholderFromFileContent{MYLIB}{mylib.py}
%% This will set MYLIB to the content of the file mylib.py
%% Tested!
\NewDocumentCommand{\robExtPlaceholderFromFileContent}{mm}{
  \ior_open:Nn \g_robExt_read_ior {#2}
  %% Put the file in l_robExt_tmp_contain_file
  \str_clear_new:N \l_robExt_tmp_contain_file
  \ior_str_map_inline:Nn \g_robExt_read_ior {
    \str_put_right:Nx \l_robExt_tmp_contain_file {\tl_to_str:N{##1}^^J}
  }
  %% sets the template name to the relative path to the file
  \str_set_eq:cN { l_robExt_placeholder_#1_str } \l_robExt_tmp_contain_file
  \robExtAddPlaceholderToList{#1}
}
\let\placeholderFromFileContent\robExtPlaceholderFromFileContent


%% Usage:
%% \placeholderPathFromContent{MYLIBPATH}{some code}
%% This will copy "some code" in the cache, and set MYLIBPATH to the name of the file in the cache like
%% MYLIBPATH = robExt-abc.py
%% Tested!
\NewDocumentCommand{\robExtPlaceholderPathFromContent}{mO{.tex}m}{
  \str_set_hash_robust:Nn \l_robExt_tmp_contain_file {#3}
  %% computes the new filename based on the md5
  \str_clear_new:N \l_robExt_tmp_filename
  \str_set:Nx \l_robExt_tmp_filename_no_prefix {\pdfmdfivesum{\l_robExt_tmp_contain_file}#2}
  %% Writes the content to the file
  \robExtCheckIfPrefixFolderExists
  \iow_open:Nx \g_robExt_write_iow {\robExtAddCachePathAndName{\l_robExt_tmp_filename_no_prefix}}
  \iow_now:NV \g_robExt_write_iow \l_robExt_tmp_contain_file
  \iow_close:N \g_robExt_write_iow
  %% sets the template name to the relative path to the file
  \str_set:cx { l_robExt_placeholder_#1_str } {\robExtPrefixFilename\l_robExt_tmp_filename_no_prefix}
  \robExtAddPlaceholderToList{#1}
}
\let\placeholderPathFromContent\robExtPlaceholderPathFromContent


%% Usage:
%% \begin{PlaceholderPathFromCode}{mylibpath}
%% def blabla():
%% \end{PlaceholderPathFromCode}
%% This will copy "some code" in the cache, and set MYLIBPATH to the name of the file in the cache like
%% MYLIBPATH = robExt-abc.py
\NewDocumentEnvironment{RobExtPlaceholderPathFromCode}{O{}m}{
  \XSIMfilewritestart*{\jobname-robExt-tmp-file-you-can-remove.tmp}
}{
  \XSIMfilewritestop
  \ior_open:Nn \g_robExt_read_ior {\jobname-robExt-tmp-file-you-can-remove.tmp}
  %% Put the file in \l_robExt_tmp_contain_file
  \str_clear_new:N \l_robExt_tmp_contain_file
  \ior_str_map_inline:Nn \g_robExt_read_ior {
    \str_gput_right:Nx \l_robExt_tmp_contain_file {\tl_to_str:N{##1}^^J}
  }
  %% computes the new filename based on the md5
  \str_clear_new:N \l_robExt_tmp_filename
  \str_set:Nx \l_robExt_tmp_filename_no_prefix {\pdfmdfivesum{\l_robExt_tmp_contain_file}#1}
  %% Writes the content to the file
  \robExtCheckIfPrefixFolderExists
  \iow_open:Nx \g_robExt_write_iow {\robExtAddCachePathAndName{\l_robExt_tmp_filename_no_prefix}}
  \iow_now:NV \g_robExt_write_iow \l_robExt_tmp_contain_file
  \iow_close:N \g_robExt_write_iow
  %% sets the template name to the relative path to the file
  \str_set:cx { l_robExt_placeholder_#2_str } {\robExtPrefixFilename\l_robExt_tmp_filename_no_prefix}
  \robExtAddPlaceholderToList{#2}
  %% Otherwise they will be lost when the environment ends
  \robExtKeepaftergroup{l_robExt_placeholders_seq}
  %% for other variable
  \robExtKeepaftergroup{l_robExt_placeholder_#2_str}
}
\let\PlaceholderPathFromCode\RobExtPlaceholderPathFromCode
\let\endPlaceholderPathFromCode\endRobExtPlaceholderPathFromCode



%%% Evaluate a string by replacing the placeholders until there is none left
%%% the result will be in \robExtResult
\cs_set:Nn \__replace_until_impossible:N {
  \str_set_eq:NN \l_robext_tmp_before \l_robExt_result_str
  \seq_map_inline:Nn \l_robExt_placeholders_seq {
    \str_replace_all:Nnv \l_robExt_result_str { ##1 } { l_robExt_placeholder_##1_str }
  }
  % We compare the result
  \str_compare:eNeTF \l_robext_tmp_before = \l_robExt_result_str {
    %\str_set_eq:cN { l_robExt_placeholder_#1_str } \l_robExt_result_str
    %\robExtAddPlaceholderToList{#1}
  }{
    % The strings are different: we restart
    \__replace_until_impossible:N { }
  }
}

\NewDocumentCommand{\robExtGetPlaceholderInResult}{O{}m}{
  \str_set_hash_robust:Nn \l_robExt_result_str { #2 }
  \__replace_until_impossible:N { }
  \tl_if_blank:nTF {#1} {} {
    % To avoid infinite recursion later and allow concatenation to a placeholder that does not exists
    % we remove the name of the placeholder at the end
    \str_remove_all:Nn \l_robExt_result_str { #1 }
    \str_set_eq:cN { l_robExt_placeholder_#1_str } \l_robExt_result_str
    \robExtAddPlaceholderToList{#1}
  }
}
\let\getPlaceholderInResult\robExtGetPlaceholderInResult

\cs_set:Nn \__replace_from_list:N {
  \str_set_eq:NN \l_robext_tmp_before \l_robExt_result_str
  \clist_map_inline:Nn {#1} {
    \str_replace_all:Nnv \l_robExt_result_str { ##1 } { l_robExt_placeholder_##1_str }
  }
  % We compare the result
  \str_compare:eNeTF \l_robext_tmp_before = \l_robExt_result_str {
    % \str_set_eq:cN { l_robExt_placeholder_#1_str } \l_robExt_result_str
    % \robExtAddPlaceholderToList{#1}
  }{
    % The strings are different: we restart
    \__replace_from_list:N { #1 }
  }
}

\NewDocumentCommand{\robExtGetPlaceholderInResultReplaceFromList}{mO{}m}{
  \str_set_hash_robust:Nn \l_robExt_result_str { #3 }
  \clist_set:Nn \l_robExt_list_placeholder_replace_clist {#1}
  \__replace_from_list:N { \l_robExt_list_placeholder_replace_clist }
  \tl_if_blank:nTF {#2} {} {
    \str_remove_all:Nn \l_robExt_result_str { #1 }
    \str_set_eq:cN { l_robExt_placeholder_#2_str } \l_robExt_result_str
    \robExtAddPlaceholderToList{#2}
  }
}
\let\getPlaceholderInResultReplaceFromList\robExtGetPlaceholderInResultReplaceFromList


% Otherwise we need latex3 
\def\robExtResult{\l_robExt_result_str}

%%%% Same version, but replaces only one (useful sometimes)
\cs_set:Nn \__replace_n_times:n {
  \str_set_eq:NN \l_robext_tmp_before \l_robExt_result_str
  \seq_map_inline:Nn \l_robExt_placeholders_seq {
    \str_replace_all:Nnv \l_robExt_result_str { ##1 } { l_robExt_placeholder_##1_str }
  }
  \int_compare:nTF { #1 > 1}{
    \__replace_n_times:n {\int_eval:n {#1-1}}
  }{}
}


\NewDocumentCommand{\robExtSetPlaceholderRec}{mm}{
  \robExtGetPlaceholderInResult[#1]{#2}
}
\let\setPlaceholderRec\robExtSetPlaceholderRec

\NewDocumentCommand{\robExtSetPlaceholderRecReplaceFromList}{mmm}{
  \robExtGetPlaceholderInResultReplaceFromList{#1}[#2]{#3}
}
\let\setPlaceholderRecReplaceFromList\robExtSetPlaceholderRecReplaceFromList


\NewDocumentCommand{\robExtGetPlaceholder}{O{}m}{
  \robExtGetPlaceholderInResult[#1]{#2}
  \l_robExt_result_str
}
\let\getPlaceholder\robExtGetPlaceholder

%%% Evaluate a string by replacing the placeholders until there is none left
%%% the result will be in \robExtResult
\NewDocumentCommand{\robExtEvalPlaceholder}{m}{
  \str_set_hash_robust:Nn \l_test_str {#1}
  \robExtGetPlaceholderInResult{#1}
  \tl_rescan:nv {} { l_robExt_result_str }
}
\let\evalPlaceholder\robExtEvalPlaceholder

%%% Evaluate a string by replacing the placeholders until there is none left
%%% the result will be in \robExtResult
\NewDocumentCommand{\robExtEvalPlaceholderReplaceNTimes}{O{1}m}{
  \str_set_hash_robust:Nn \l_test_str {#2}
  \robExtGetPlaceholderInResultReplaceNTimes[#1]{#2}
  \tl_rescan:nv {} { l_robExt_result_str }
}
\let\evalPlaceholderReplaceNTimes\robExtEvalPlaceholderReplaceNTimes

%%% Evaluate a string by replacing the placeholders until there is none left
%%% the result will be in \robExtResult
\NewDocumentCommand{\robExtEvalPlaceholderReplaceFromList}{mm}{
  \str_set_hash_robust:Nn \l_test_str {#2}
  \robExtGetPlaceholderInResultReplaceFromList{#1}{#2}
  \tl_rescan:nv {} { l_robExt_result_str }
}
\let\evalPlaceholderReplaceFromList\robExtEvalPlaceholderReplaceFromList


\NewDocumentCommand{\robExtEvalPlaceholderInplace}{m}{
  \robExtGetPlaceholderInResult{#1}
  \tl_set_rescan:Nnx \l_robExt_tmp_tl  {} { \l_robExt_result_str }
  \str_set_hash_robust:cx { l_robExt_placeholder_#1_str } { \l_robExt_tmp_tl }
}
\let\evalPlaceholderInplace\robExtEvalPlaceholderInplace

% Sometimes two symbols ## (with a different catcode than a normal hash) are added instead of a single #
\NewDocumentCommand{\robExtPlaceholderDoubleNumberHashesInplace}{m}{
  \str_replace_all:cxx { l_robExt_placeholder_#1_str } { \c_hash_str } { \c_hash_str \c_hash_str }
}
\let\placeholderDoubleNumberHashesInplace\robExtPlaceholderDoubleNumberHashesInplace

\NewDocumentCommand{\robExtPlaceholderHalveNumberHashesInplace}{m}{
  \str_replace_all:cxx { l_robExt_placeholder_#1_str } { \c_hash_str \c_hash_str } { \c_hash_str }
}
\let\placeholderHalveNumberHashesInplace\robExtPlaceholderHalveNumberHashesInplace

\NewDocumentCommand{\robExtPlaceholderReplaceInplace}{mmm}{
  \str_replace_all:cnn { l_robExt_placeholder_#1_str } { #2 } { #3 }
}
\let\placeholderReplaceInplace\robExtPlaceholderReplaceInplace

\NewDocumentCommand{\robExtPlaceholderReplaceInplaceEval}{mmm}{
  \str_replace_all:cxx { l_robExt_placeholder_#1_str } { #2 } { #3 }
}
\let\placeholderReplaceInplaceEval\robExtPlaceholderReplaceInplaceEval


%%%%%%%%%%%%%%%%%%%%%%%%%%%%%%%%%%%%%
%%%%%%%%%%% Dependencies %%%%%%%%%%%%
%%%%%%%%%%%%%%%%%%%%%%%%%%%%%%%%%%%%%
\NewDocumentCommand{\robExtResetDependencies}{m}{
  \seq_clear:N \l_robExt_dependencies
}

\NewDocumentCommand{\robExtAddDependency}{m}{
  \seq_put_left:Nx \l_robExt_dependencies {#1}
}

\NewDocumentCommand{\robExtDebugDependency}{}{
  \show\l_robExt_dependencies
}


%%%%%%%%%%%%%%%%%%%%%%%%%%%%%%%%%%%%%%%%
%%%%%%%%%%% Externalization %%%%%%%%%%%%
%%%%%%%%%%%%%%%%%%%%%%%%%%%%%%%%%%%%%%%%
%% The idea is to populate a placeholder called __ROBEXT_TEMPLATE__ that will contain the file to generate
%% together with a placeholder called __ROBEXT_COMPILATION_COMMAND__ that will contain the command to compile the file

\NewDocumentCommand{\robExtSetCompilationCommand}{m}{
  \robExtSetPlaceholder{__ROBEXT_COMPILATION_COMMAND__} {#1}
}

\NewDocumentCommand{\robExtAddArgumentToCompilationCommand}{m}{
  \robExtSetPlaceholderRec{__ROBEXT_COMPILATION_COMMAND__} {__ROBEXT_COMPILATION_COMMAND__ ~ "#1"}
}


%% Alias of robExtFinalFile to \robExtSourceFile, as I don't like anymore the name I chose
%\def\robExtSourceFile{\robExtFinalFile}

%%% \l_robExt_final_file must contain before calling this function the content of the final file.
%%% \l_robExt_dependencies must contain the extensions (list).
%%% \l_robExt_currentCompilationCommand contains the compilation command to use.
%%% Note that we do note parse them as input to allow more flexibility on the way the user
%%% defines them, and to limit issues with expansion.
\NewDocumentCommand{\robExtWriteFile}{m}{
  %%% First we get all dependencies stored in \l_robExt_dependencies to create a csv-like file:
  \str_clear:N \l_robExt_dependencies_mdfive
  %% Make sure to remove these placeholders as they should not be replaced.
  %% Not that we cannot just give them their final value here, as it cannot yet be determined without
  %% first computing the md5 hash.
  \robExtRemovePlaceholder{__ROBEXT_SOURCE_FILE__}
  \robExtRemovePlaceholder{__ROBEXT_OUTPUT_PDF__}
  \robExtRemovePlaceholder{__ROBEXT_OUTPUT_PREFIX__}
  \setPlaceholder{__ROBEXT_WAY_BACK__}{\robExtCacheFolderWayBack}
  \robExtEvalPlaceholderInplace{__ROBEXT_WAY_BACK__}
  \setPlaceholder{__ROBEXT_CACHE_FOLDER__}{\robExtCacheFolder}
  \robExtEvalPlaceholderInplace{__ROBEXT_CACHE_FOLDER__}
  %%% We rescan the string in order to evaluate stuff like \myframes into "12,45,56".
  \robExtGetPlaceholderInResult{__ROBEXT_COMPILATION_COMMAND__}
  \ifdefined\robExtDoNotRescanFirstTime
    \str_set_eq:NN \l_robExt_currentCompilationCommand \l_robExt_result_str
  \else
    \tl_set_rescan:Nnx \l_robExt_currentCompilationCommand  {} { \l_robExt_result_str }
  \fi%
  %% We get the template
  \robExtGetPlaceholderInResult{__ROBEXT_TEMPLATE__}
  \str_set_eq:NN \l_robExt_final_file_minus_hash_str \l_robExt_result_str
  % We first add on the first line the compilation command, and on the second line the template file.
  \str_set:Nx \l_robExt_dependencies_mdfive {command,\l_robExt_currentCompilationCommand^^J\pdfmdfivesum{\l_robExt_final_file_minus_hash_str ^^J},^^J} %% ^^J is a newline: LaTeX will automatically add a new line when writing the file
  \seq_map_inline:Nn \l_robExt_dependencies {
    \str_put_right:Nx \l_robExt_dependencies_mdfive {\file_mdfive_hash:n{##1},##1^^J} %% ^^J is a newline
  }
  %%
  %% Compute the final hash (the hash of all dependencies, including the current picture that is on the first line):
  %% The last newline is needed as the write operation automatically adds a newline.
  \tl_set:Nx \robExtFinalHash {\pdfmdfivesum{\l_robExt_dependencies_mdfive^^J}}
  %% We add the figure in the list of files.
  \iow_now:Nx \g_robExt_write_list_all_figures {\robExtAddPrefixName{\robExtFinalHash.tex}^^J}
  %% We can now set the placeholders, and recompute the final value of the file:
  \robExtPlaceholderFromContent{__ROBEXT_SOURCE_FILE__}{\robExtAddPrefixName{\robExtFinalHash.tex}}
  \robExtEvalPlaceholderInplace{__ROBEXT_SOURCE_FILE__}
  \robExtPlaceholderFromContent{__ROBEXT_OUTPUT_PDF__}{\robExtAddPrefixName{\robExtFinalHash.pdf}}
  \robExtEvalPlaceholderInplace{__ROBEXT_OUTPUT_PDF__}
  \robExtPlaceholderFromContent{__ROBEXT_OUTPUT_PREFIX__}{\robExtAddPrefixName{\robExtFinalHash}}
  \robExtEvalPlaceholderInplace{__ROBEXT_OUTPUT_PREFIX__}
  \robExtGetPlaceholderInResult{__ROBEXT_TEMPLATE__}
  \str_set_eq:NN \l_robExt_final_file_str \l_robExt_result_str
  \file_if_exist:xTF{\robExtAddCachePathAndName{\robExtFinalHash.tex}}{
    \message{The\space file\space \robExtAddCachePathAndName{\robExtFinalHash.tex} \space already\space exists.^^J}
  }{
    \str_if_eq:VnTF { \l_robExt_final_file_str }{__ROBEXT_TEMPLATE__} {
      \PackageError{robExt}{You ~ are ~ writing ~ __ROBEXT_TEMPLATE__ ~ to ~ your ~ file: ~ seems ~ like ~ you ~ forgot ~ to ~ define ~ your ~ template ~ or ~ set ~ a ~ preset}{}
    }{
      % Check if the output directory exists
      \robExtCheckIfPrefixFolderExists
      \iow_open:Nx \g_robExt_write_iow {\robExtAddCachePathAndName{\robExtFinalHash.deps}}
      \iow_now:NV \g_robExt_write_iow \l_robExt_dependencies_mdfive
      \iow_close:N \g_robExt_write_iow
      %% Save the final file:
      \iow_open:Nx \g_robExt_write_iow {\robExtAddCachePathAndName{\robExtFinalHash.tex}}
      \iow_now:NV \g_robExt_write_iow \l_robExt_final_file_str
      \iow_close:N \g_robExt_write_iow
      \message{Source ~ saved ~ in ~ \robExtAddCachePathAndName{\robExtFinalHash.tex}.}
    }
  }
}

% https://tex.stackexchange.com/questions/133324/shell-escape-with-latex-3
% We need shell escape to work (but it's enabled by default on overleaf!)
% Think about the number of compilations.
\NewDocumentCommand{\robExtCompileFile}{m}{
  \file_if_exist:xTF{\robExtAddCachePathAndName{\robExtFinalHash.pdf}}{
    \message{No ~ need ~ to ~ recompile ~ \robExtAddCachePathAndName{\robExtFinalHash.pdf}^^J}
  }{
    \robExtGetPlaceholderInResult{__ROBEXT_COMPILATION_COMMAND__}
    \ifdefined\robExtDoNotRescanSecondTime
      \str_set_eq:NN \l_robExt_finalCompilationCommand \l_robExt_result_str
    \else
      \tl_set_rescan:Nnx \l_robExt_finalCompilationCommand  {} { \l_robExt_result_str }
    \fi%
    % Make sure this command is run from the cache folder
    \ifdefined\robExtCacheFolder
      \str_put_left:Nx \l_robExt_finalCompilationCommand {cd ~ \robExtCacheFolder \space && ~ }
    \fi
    \ifdefined\robExtManualMode
      \message{[robExt] Manual mode enabled: please, manually compile the images using \l_robExt_finalCompilationCommand or run 'bash \jobname-\robExtAddPrefixName{compile-missing-figures.sh}'.}
      \iow_now:Nx \g_robExt_write_manually_compile_all_missing_figures {\l_robExt_finalCompilationCommand^^J}
    \else
      \bool_if:nTF { \sys_if_shell_unrestricted_p: || \cs_if_exist_p:N {\robExtForceCompilation} }
      {
        \message{[robExt] We ~ will ~ start ~ the ~ compilation using: ~ \l_robExt_finalCompilationCommand.}
        \sys_shell_now:x {\l_robExt_finalCompilationCommand} % The ~ are used in ExplSyntaxOn to add space
      }{
        \ifdefined\robExtFallbackManualMode
          \message{[robExt] Fallback to manual mode: please, manually compile the images using \l_robExt_finalCompilationCommand or run 'bash \jobname-\robExtAddPrefixName{compile-missing-figures.sh}'.}
          \iow_now:Nx \g_robExt_write_manually_compile_all_missing_figures {\l_robExt_finalCompilationCommand^^J}
        \else
          \PackageError{robExt}{You ~ need ~ to ~ compile ~ with ~ shell-escape ~ as ~ in: ~ "pdflatex ~ -shell-escape ~ yourfile.tex" ~ to ~ be ~ able ~ to ~ compile ~ automatically ~ the ~ figures}{}
        \fi
      }
    \fi
  }
}

\def\robExtIncludeGraphicsArgs{}
%%% This command is not meant to be called by the end user. It will be called after the compilation to include
%%% the compiled file back into the original file.
\NewDocumentCommand{\robExtIncludeFile}{m}{%
  \ifdefined\robExtIncludeCommandAdvanced%
    \robExtIncludeCommandAdvanced%
  \else%
    {%
      \file_if_exist:xTF{\robExtAddCachePathAndName{\robExtFinalHash.pdf}}{%
        \file_if_exist:xTF{\robExtAddCachePathAndName{\robExtFinalHash-out.tex}}{%
          \kern0pt%Without the kern, the next unskip would eat spaces before... and we don't want that. See also
          % https://tex.stackexchange.com/questions/104034/when-is-it-good-practice-to-use-unskip
          \input{\robExtAddCachePathAndName{\robExtFinalHash-out.tex}}\unskip% Otherwise if the file contains space it will be added here.
        }{}%
        \ifdefined\robExtIncludeCommand%
          \robExtIncludeCommand%
        \else%
          \evalPlaceholder{%
            \ifdefined\robExtDepth%
              \raisebox{-\robExtDepth}{%
                \includegraphics[__ROBEXT_INCLUDEGRAPHICS_OPTIONS__]{%
                  __ROBEXT_INCLUDEGRAPHICS_FILE__%
                  }}%
            \else%
              \includegraphics[__ROBEXT_INCLUDEGRAPHICS_OPTIONS__]{%
                \robExtAddCachePathAndName{\robExtFinalHash.pdf}}%
            }%
          \fi%
        \fi%
      }{
        \ifdefined\robExtManualMode
          \framebox[\linewidth]
          {
            \begin{minipage}{\linewidth}
              \textbf{Draft ~ mode: ~ either ~ compile ~ with ~ \texttt{-shell-escape} ~ or ~ compile:\newline \texttt{\robExtAddCachePathAndName{\robExtFinalHash.tex}}\newline via ~ \newline \texttt{\l_robExt_finalCompilationCommand}\newline or ~ call ~ \newline\texttt{bash ~ \jobname-\robExtAddPrefixName{compile-missing-figures.sh}}\newline to ~ compile ~ all ~ missing ~ figures.}
            \end{minipage}
          }
          %\fbox{\textbf{Draft ~ Mode: ~ you ~ are ~ in ~ manual ~ mode:}\par\textbf{ ~ please ~ compile ~ \robExtAddCachePathAndName{\robExtFinalHash.tex} ~ or ~ use ~ \ifdefined\robExtCacheFolder cd \robExtCacheFolder; \fi bash ~ \jobname-\robExtAddPrefixName{compile-missing-figures.sh}}}
          \message{[robExt] ~ You ~ are ~ in ~ manual ~ mode: ~ please ~ compile ~ yourself ~ \robExtAddCachePathAndName{\robExtFinalHash.tex} ~ or ~ use ~ the ~ bash ~ \jobname-\robExtAddPrefixName{compile-missing-figures.sh}}
        \else
          \ifdefined\robExtFallbackManualMode
            \framebox[\linewidth]
            {
              \begin{minipage}{\linewidth}
                \textbf{Falling ~ back ~ to ~ draft ~ mode: ~ either ~ compile ~ with ~ \texttt{-shell-escape} ~ or ~ compile:\newline \texttt{\robExtAddCachePathAndName{\robExtFinalHash.tex}}\newline via ~ \newline \texttt{\l_robExt_finalCompilationCommand}\newline or ~ call ~ \newline\texttt{bash ~ \jobname-\robExtAddPrefixName{compile-missing-figures.sh}}\newline to ~ compile ~ all ~ missing ~ figures.}
              \end{minipage}
            }
            % \fbox{\textbf{Draft ~ Mode: ~ you ~ are ~ in ~ manual ~ mode:}\par\textbf{ ~ please ~ compile ~ \robExtAddCachePathAndName{\robExtFinalHash.tex} ~ or ~ use ~ \ifdefined\robExtCacheFolder cd \robExtCacheFolder; \fi bash ~ \jobname-\robExtAddPrefixName{compile-missing-figures.sh}}}
            \message{[robExt] ~ You ~ are ~ falling ~ back ~ to ~ manual ~ mode: ~ please ~ compile ~ yourself ~ \robExtAddCachePathAndName{\robExtFinalHash.tex} ~ or ~ use ~ the ~ bash ~ \jobname-\robExtAddPrefixName{compile-missing-figures.sh}}
          \else
            \PackageError{robExt}{For ~ an ~ unknown ~ reason ~ the ~ pdf ~ file ~ \robExtAddCachePathAndName{\robExtFinalHash.pdf} ~ is ~ not ~ present. ~ The ~ compilation ~ command ~ certainly ~ failed, ~ see ~ logs ~ above.}{}
          \fi
        \fi
      }
    }%
  \fi%
}

%%%%%%%%%%%%%%%%%%%%%%%%%%%%%%%%%%
%%% Disabling externalization
%%%%%%%%%%%%%%%%%%%%%%%%%%%%%%%%%%

% It seems that when we disable externalization on \tikz, \tikz internally call \tikzpicture in a
% weird way (certainly some tikz magic), and as a result it does not manage to grab the end of the CacheMe
% environment. For this reason, by default, |disable externalization| will disable **all** commands

\seq_clear_new:N \l_commands_to_reset_seq % List of commands to reset by default when disable externalization is set

\NewDocumentCommand{\robExtAddToCommandResetList}{m}{
  \seq_put_right:Nn \l_commands_to_reset_seq {#1}
}

\NewDocumentCommand{\robExtSetCommandResetList}{m}{
  \seq_set_from_clist:Nn \l_commands_to_reset_seq {#1}
}

\seq_clear_new:N \l_environments_to_reset_seq % List of environments to reset by default when disable externalization is set

\NewDocumentCommand{\robExtAddToEnvironmentResetList}{m}{
  \seq_put_right:Nn \l_environments_to_reset_seq {#1}
}

\NewDocumentCommand{\robExtSetEnvironmentResetList}{m}{
  \seq_set_from_clist:Nn \l_environments_to_reset_seq {#1}
}

\NewDocumentCommand{\robExtDisableTikzpictureOverwrite}{}{%
  \ifdefined\robExtTikzPictureOrig%
    \let\tikzpicture\robExtTikzPictureOrig%
    \let\endtikzpicture\endrobExtTikzPictureOrig%
  \fi%
  \ifdefined\robExtEnvironmentOrigName%
    \expanded{\noexpand\DeclareEnvironmentCopy{\robExtEnvironmentOrigName}{robExtEnvironmentOrig\robExtEnvironmentOrigName}}%
  \fi%
  \ifdefined\robExtCommandOrigName%
    \expandafter\DeclareCommandCopy\csname \robExtCommandOrigName\expandafter\endcsname\csname robExtCommandOrig\robExtCommandOrigName\endcsname%
  \fi%
  \ifdefined\robExtDoNotResetAllCommands\else%
    \seq_map_inline:Nn \l_commands_to_reset_seq {
      \cs_if_exist:cTF { robExtCommandOrig##1 } {
        \expandafter\DeclareCommandCopy\csname ##1\expandafter\endcsname\csname robExtCommandOrig##1\endcsname%
      } {}
    }
    \seq_map_inline:Nn \l_environments_to_reset_seq {
      \cs_if_exist:cTF { robExtEnvironmentOrig##1 } {
        \expanded{\noexpand\DeclareEnvironmentCopy{##1}{robExtEnvironmentOrig##1}}
      } {}
    }
  \fi%
}


\ExplSyntaxOff

%%%%%%%%%%%%%%%%%%%%%%%%%%%%%%%%%%
%%% Interface
%%%%%%%%%%%%%%%%%%%%%%%%%%%%%%%%%%
% We create interface into pgfkeys in order to allow easier creation of content via style
\pgfkeys{
  /robExt/.cd,
  % We create a default style that will be loaded (mostly for the user)
  default style/.style={},
  %%%%%%%%%%%%%%%%%%%%%%%%%%%%%%%%%
  %%% Code to create new styles %%%
  %%%%%%%%%%%%%%%%%%%%%%%%%%%%%%%%%
  % The advantage of this over .append style is that you do not need to double the number of hashes
  % don't know if there is a better solution.
  % https://tex.stackexchange.com/questions/695432/latex3-latex-doubles-the-number-of-hashes-when-storing-them-in-string/695461
  add to preset/.code 2 args={%
    \robExtStrSetDoubleHash{\robExtTmpStr}{#2}
    % Sadly, \expanded{\noexpand } does not work, as I get extra {} around the def, creating a group
    % so the simpler seems to use this library ^^
    \robExtPlaceholderFromString{__ROBEXT_TMP__}{\robExtTmpStr}%
    \robExtEvalPlaceholderReplaceFromList{__ROBEXT_TMP__}{%
      \pgfkeys{%
        /robExt/.cd,
        #1/.append style={__ROBEXT_TMP__},%
      }%
    }%
    \robExtRemovePlaceholder{__ROBEXT_TMP__}% let us clean our variables
  },
  new preset/.code 2 args={%
    \robExtStrSetDoubleHash{\robExtTmpStr}{#2}%
    % Sadly, \expanded{\noexpand } does not work, as I get extra {} around the def, creating a group
    % so the simpler seems to use this library ^^
    \robExtPlaceholderFromString{__ROBEXT_TMP__}{\robExtTmpStr}%
    % \robExtShowPlaceholder*{__ROBEXT_TMP__}
    \robExtEvalPlaceholderReplaceFromList{__ROBEXT_TMP__}{%
      \pgfkeys{%
        /robExt/.cd,
        #1/.style={__ROBEXT_TMP__},%
      }%
    }%
    \robExtRemovePlaceholder{__ROBEXT_TMP__}% let us clean our variables
  },
  % reset the main content to the main content orig
  % Use case: e.g. tikz will wrap automatically its code in \begin{tikzpicture}
  % but this should not be set in a command. So we can use \cacheMe{tikz, in command}{\tikz\node{A};}
  in command/.style={
    set placeholder={__ROBEXT_MAIN_CONTENT__}{__ROBEXT_MAIN_CONTENT_ORIG__},
  },
  %%%%%%%%%%%%%%%%%%%%%%%%%%%%%%%%%%%%%%%%
  %%% Interface to change placeholders %%%
  %%%%%%%%%%%%%%%%%%%%%%%%%%%%%%%%%%%%%%%%
  remove placeholder/.code={\robExtRemovePlaceholder{#1}},
  remove placeholders/.style={
    remove placeholder/.list={#1},
  },
  set placeholder/.code 2 args={\robExtSetPlaceholder{#1}{#2}},
  set main content/.style={
    set placeholder={__ROBEXT_MAIN_CONTENT_ORIG__}{#1}
  },
  show placeholder/.code={\robExtShowPlaceholder*{#1}},
  copy placeholder/.code 2 args={\robExtCopyPlaceholder{#1}{#2}},
  set placeholder rec/.code 2 args={\robExtSetPlaceholderRec{#1}{#2}},
  set placeholder rec replace from list/.code n args={3}{\robExtSetPlaceholderRecReplaceFromList{#1}{#2}{#3}},
  set placeholder eval/.code 2 args={\robExtSetPlaceholderRec{#1}{#2}\robExtEvalPlaceholderInplace{#1}},
  set placeholder eval replace from list/.code n args={3}{\robExtSetPlaceholderRecReplaceFromList{#1}{#2}{#3}\robExtEvalPlaceholderInplace{#2}},
  eval placeholder/.code={\robExtEvalPlaceholder{#1}},
  eval placeholder replace from list/.code 2 args={\robExtEvalPlaceholderReplaceFromList{#1}{#2}},
  set placeholder from content/.code 2 args={\robExtPlaceholderFromContent{#1}{#2}},
  add to placeholder/.code 2 args={\robExtAddToPlaceholder{#1}{#2}},
  add to placeholder no space/.code 2 args={\robExtAddToPlaceholder*{#1}{#2}},
  add before placeholder/.code 2 args={\robExtAddBeforePlaceholder{#1}{#2}},
  add before placeholder no space/.code 2 args={\robExtAddBeforePlaceholder*{#1}{#2}},
  set placeholder path from filename/.code 2 args={\robExtPlaceholderPathFromFilename{#1}{#2}},
  set placeholder from file content/.code 2 args={\robExtPlaceholderFromFileContent{#1}{#2}},
  set placeholder path from content/.code n args={3}{\robExtPlaceholderPathFromContent{#1}[#3]{#2}},
  eval placeholder in place/.code={\robExtEvalPlaceholderInplace{#1}},
  placeholder halve number hashes in place/.code={\robExtPlaceholderHalveNumberHashesInplace{#1}},
  placeholder double number hashes in place/.code={\robExtPlaceholderDoubleNumberHashesInplace{#1}},
  placeholder replace in place/.code n args={3}{\robExtPlaceholderReplaceInplace{#1}{#2}{#3}},
  placeholder replace in place eval/.code n args={3}{\robExtPlaceholderReplaceInplaceEval{#1}{#2}{#3}},
  % Interface to set template
  set template/.style={
    set placeholder={__ROBEXT_TEMPLATE__}{#1},
  },
  %%%%%%%%%%%%%%%%%%%%%%%%%%%%%% 
  %%% Configure dependencies %%%
  %%%%%%%%%%%%%%%%%%%%%%%%%%%%%% 
  %%% Auxiliary command:
  dependenciesList/.code={\robExtAddDependency{#1}},
  % Usage like: dependencies={input_externalize.tex,input_b.tex}
  % They should be relative to the main file when using the subfolder option.
  dependencies/.style={
    /utils/exec={\robExtResetDependencies{}},
    dependenciesList/.list={#1}
  },
  add dependencies/.style={
    dependenciesList/.list={#1}
  },
  reset dependencies/.code={\robExtResetDependencies{}},
  %%%%%%%%%%%%%%%%%%%%%%%%%%% 
  %%% Compilation command %%%
  %%%%%%%%%%%%%%%%%%%%%%%%%%%
  % People might want to force the compilation even if shell_unrestricted is false, notably if they
  % allow commands like mkdir/cd/pdflatex to run in restricted mode.
  force compilation/.code={\def\robExtForceCompilation{}},
  do not force compilation/.code={\let\robExtForceCompilation\undefined},
  set compilation command/.code={\robExtSetCompilationCommand{#1}},
  add argument to compilation command/.code={\robExtAddArgumentToCompilationCommand{#1}},
  add arguments to compilation command/.style={
    add argument to compilation command/.list={#1}
  },
  % This adds arguments like add key value to compilation command={mykey=myvalue} will add to the
  % compilation command two arguments: "mykey" "myvalue"
  % This is useful for scripts that are called like myscript key1 arg1 key2 arg2 key3 arg3, which is a
  % simple way to pass multiple arguments to a script like a python script
  add key value argument to compilation command/.code args={#1=#2}{\robExtAddArgumentToCompilationCommand{#1}\robExtAddArgumentToCompilationCommand{#2}},
  add key and file argument to compilation command aux/.style args={#1=#2}{
    add key value argument to compilation command={{#1}={\ifdefined\robExtCacheFolderWayBack\robExtCacheFolderWayBack\fi#2}},
  },
  add key and file argument to compilation command/.style={
    add key and file argument to compilation command aux/.list={#1},
    add dependencies={#1},
  },
  %%%%%%%%%%%%%%%%%%%%%%%%% 
  %%% Inclusion command %%%
  %%%%%%%%%%%%%%%%%%%%%%%%% 
  %%% Configure the command to include the compiled file back into the main file
  % By default, include command does a bit of logic before running the actual command, notably to
  % input the -out.tex file in order to pass information from the compiled file to the current file.
  % If you want to do everything by yourself, use:
  custom include command advanced/.code={\def\robExtIncludeCommandAdvanced{#1}},
  % The default include command includes the pdf, making sure it is raised depending on its depth,
  % but you can override it:
  custom include command/.code={\def\robExtIncludeCommand{#1}},
  %% Use this when we do not want to include anything (e.g. the video will be processed later in the chain):
  do not include pdf/.style={
    custom include command={}%
  },
  %% If you do or do not want to ask latex to run the compilation commands itself (for instance for security
  %% reasons, you can use these commands and run the command manually later):
  enable manual mode/.code={\def\robExtManualMode{}},
  disable manual mode/.code={\let\robExtManualMode\undefined},
  enable fallback to manual mode/.code={\def\robExtFallbackManualMode{}},
  disable fallback to manual mode/.code={\let\robExtFallbackManualMode\undefined},
  %% Arguments to include graphics
  include graphics args/.code={\def\robExtIncludeGraphicsArgs{#1}},
  %% The role of this command is to set \l_robExt_result_str, that will contain the final string.
  %%%%%%%%%%%%%%%%%%%%%%%%%%%%%%%%%% 
  %%% Configuration of the cache %%%
  %%%%%%%%%%%%%%%%%%%%%%%%%%%%%%%%%% 
  %% Configure the prefix (default to "robExt-")
  set filename prefix/.code={\def\robExtPrefixFilename{#1}},
  % first argument is subfolder, second is how to get from subfolder to the folder containing the source:
  % set subfolder and way back={robustExternal/}{../}
  % synonyme, "cache folder" is prefered over ""
  set subfolder and way back/.code 2 args={\def\robExtCacheFolder{#1}\def\robExtCacheFolderWayBack{#2}},
  set cache folder and way back/.code 2 args={\def\robExtCacheFolder{#1}\def\robExtCacheFolderWayBack{#2}},
  no cache folder/.code={\let\robExtCacheFolder\undefined\def\robExtCacheFolderWayBack{}},
  % By default we put everything in robustExternalize
  % Change this before starting to cache any library, and if you change it mid-document, be aware
  % that you will not be able to refer to elements in the old folder.
  set subfolder and way back={robustExternalize/}{../},
  %%%%%%%%%%%%%%%%%%%%%%%%%%%%%%% 
  %%% Disable externalization %%%
  %%%%%%%%%%%%%%%%%%%%%%%%%%%%%%% 
  %% Note: this does not work reliably for now
  %% TODO: fix this!
  disable externalization/.code={\def\robExtDisableExternalization{}},
  disable externalization now/.code={\robExtDisableTikzpictureOverwrite\def\robExtDisableExternalization{}},
  enable externalization/.code={\let\robExtDisableExternalization\undefined},
  % Useful to wrap, for instance, text
  command if no externalization/.code={\robExtDisableTikzpictureOverwrite\evalPlaceholder{__ROBEXT_MAIN_CONTENT__}},
  print verbatim if no externalization/.style={
    command if no externalization/.code={%
      \robExtPrintPlaceholder{__ROBEXT_MAIN_CONTENT__}%
    },
  },
  %%%%%%%%%%%%%%%%%%%%%%%%%%%%%%%%%%%%%%% 
  %%% Run code before/after inclusion %%%
  %%%%%%%%%%%%%%%%%%%%%%%%%%%%%%%%%%%%%%% 
  %%% todo: make sure that commands can be added instead of replaced
  execute before each externalization/.code={\def\robExtExecuteBefore{#1}},
  execute after each externalization/.code={\def\robExtExecuteAfter{#1}},
  %%%%%%%%%%%%%%%%%%%%%%%%%%%%%%%%%%%%%%%%%%%%%%%%%%%%%%% 
  %%% Get the name of the produced file for later use %%%
  %%%%%%%%%%%%%%%%%%%%%%%%%%%%%%%%%%%%%%%%%%%%%%%%%%%%%%% 
  %%% Here, we provide a way to put the prefixed name into a new global macro
  %%% Use like 'name output=VideoA'. This creates a few macros like:
  %%% \blenderpointNamedOutputFilenameVideoA containing thehashthatisusedforthename
  %%% \blenderpointNamedOutputPrVideoA containing thehashthatisusedforthename
  %% See \robExtGetNamedOutputFilename to get them with \robExtGetNamedOutputFullPath
  % name output/.style={name output with ext={#1}{.pdf}},
  % name input/.style={name output with ext={#1}{.tex}},
  % % like name output but adds the extension like name output with ext={VideoA}{.mp4}
  % name output with ext/.code 2 args={%
  %   \robExtSetPlaceholder{__#1__}{\robExtPrefixFilename\robExtFinalHash#2}
  %   \robExtEvalPlaceholderInplace{__#1__}
  %   \robExtSetPlaceholder{__#1_FULL_PATH__}{\robExtAddCachePathAndName{\robExtFinalHash#2}}
  %   \robExtEvalPlaceholderInplace{__#1_FULL PATH__}
  % },
  name output/.code={%
    \def\robExtExecuteNamedOutput{%
      \expandafter\xdef\csname #1\endcsname{\robExtPrefixFilename\robExtFinalHash}%
      \expandafter\xdef\csname #1InCache\endcsname{\robExtAddCachePathAndName{\robExtFinalHash}}%
    }%
  },
  %% Todo: for this to work, be will need to make the definition either global (not trivial, need to define
  %% a new list of global placeholders), or at least go past the groups
}

% Not really made for the end user
% It assumes that __ROBEXT_COMPILATION_COMMAND__ and __ROBEXT_TEMPLATE__ is set
\NewDocumentCommand{\robExtEvaluateCompileAndInclude}{}{%
  \ifdefined\robExtDisableExternalization%
    \pgfkeys{
      /robExt/.cd,
      command if no externalization
    }%
  \else%
    \ifdefined\robExtExecuteBefore\robExtExecuteBefore\fi%
    \robExtWriteFile{}%
    \robExtCompileFile{}%
    \robExtIncludeFile{}%
    \ifdefined\robExtExecuteNamedOutput\robExtExecuteNamedOutput\fi%
    \ifdefined\robExtExecuteAfter\robExtExecuteAfter\fi%
  \fi%
}

%% #1: Arguments, #2: content to externalize
\NewDocumentCommand{\robExtCacheMe}{O{}m}{%
  {% Group
    \pgfkeys{%
      /robExt/.cd,
      set placeholder={__ROBEXT_MAIN_CONTENT_ORIG__}{#2},
      default style,
      #1,
    }%
    \robExtEvaluateCompileAndInclude%
  }%
}
\let\cacheMe\robExtCacheMe

%% #1: Arguments, #2: content to externalize
\NewDocumentEnvironment{RobExtCacheMe}{m+b}{%
  \robExtCacheMe[#1]{#2}%
}{}
\let\CacheMe\RobExtCacheMe
\let\endCacheMe\endRobExtCacheMe

\NewDocumentEnvironment{RobExtCacheMeCode}{m}{%
  \RobExtPlaceholderFromCode{__ROBEXT_MAIN_CONTENT_ORIG__}%
}{%
  \endRobExtPlaceholderFromCode%
  \pgfkeys{%
    /robExt/.cd,
    #1,
  }%
  \robExtEvaluateCompileAndInclude%
}
\let\CacheMeCode\RobExtCacheMeCode
\let\endCacheMeCode\endRobExtCacheMeCode

\NewDocumentEnvironment{RobExtCacheMeNoContent}{+b}{%
  \robExtCacheMe[#1]{}%
}{}
\let\CacheMeNoContent\RobExtCacheMeNoContent
\let\endCacheMeNoContent\endRobExtCacheMeNoContent

\NewDocumentCommand{\robExtConfigure}{m}{%
  \pgfkeys{
    /robExt/.cd,#1%Do not add a space before the #1!
  }%
}



%%%%%%%%%%%%%%%%%%%%%%%%%%%%%%%%%%%%%%%%
%%%%%%%%%%% Default presets %%%%%%%%%%%%
%%%%%%%%%%%%%%%%%%%%%%%%%%%%%%%%%%%%%%%%
%% We create here a few presets and placeholders useful later

%%%%  Available in all styles
\robExtConfigure{
  set includegraphics options/.style={
    set placeholder={__ROBEXT_INCLUDEGRAPHICS_OPTIONS__}{#1},
  },
  add to includegraphics options/.style={
    add to placeholder no space={__ROBEXT_INCLUDEGRAPHICS_OPTIONS__}{,#1},
  },
  set placeholder={__ROBEXT_VERBATIM_COMMAND__}{\verbatiminput},
  % We expect the program to write in __ROBEXT_OUTPUT_PREFIX__-out.txt
  verbatim output/.style={
    custom include command={%
      \evalPlaceholder{%
        __ROBEXT_VERBATIM_COMMAND__{__ROBEXT_CACHE_FOLDER____ROBEXT_OUTPUT_PREFIX__-out.txt}%
      }%
    },
  },
  % Mostly for debugging purpose
  print command and source/.style={
    enable manual mode,
    custom include command advanced={%
      \evalPlaceholder{%
        Command: (run in folder \texttt{__ROBEXT_CACHE_FOLDER__})
        \robExtPrintPlaceholder{__ROBEXT_COMPILATION_COMMAND__}
        Dependencies:
        \verbatiminput{__ROBEXT_CACHE_FOLDER____ROBEXT_OUTPUT_PREFIX__.deps}%
        Source (in \texttt{__ROBEXT_CACHE_FOLDER____ROBEXT_OUTPUT_PREFIX__.tex}):
        \verbatiminput{__ROBEXT_CACHE_FOLDER____ROBEXT_OUTPUT_PREFIX__.tex}%
      }%
    },
  },
  debug/.style={
    print command and source
  },
}


%%%% Generic placeholders, practical to escape stuff
\ExplSyntaxOn
\robExtPlaceholderFromString{__ROBEXT_LEFT_BRACE__}{\c_left_brace_str}
\robExtPlaceholderFromString{__ROBEXT_RIGHT_BRACE__}{\c_right_brace_str}
\robExtPlaceholderFromString{__ROBEXT_BACKSLASH__}{\c_backslash_str}
\robExtPlaceholderFromString{__ROBEXT_HASH__}{\c_hash_str}
\robExtPlaceholderFromString{__ROBEXT_PERCENT__}{\c_percent_str}
\robExtPlaceholderFromString{__ROBEXT_UNDERSCORE__}{\c_underscore_str}
\ExplSyntaxOff

% This additional level of indirection is made to allow an easier wrapping
% of \begin{tikzpicture} ... \end{tikzpicture} for instance.
% The original behavior (modify __ROBEXT_MAIN_CONTENT__ directly)
% was not really practical as if you use both |tikz| and |\cacheCommand|, it would wrap
% the environment twice.
\setPlaceholder{__ROBEXT_MAIN_CONTENT__}{__ROBEXT_MAIN_CONTENT_ORIG__}


%%%% For LaTeX codes

\setPlaceholder{__ROBEXT_LATEX_OPTIONS__}{}
\setPlaceholder{__ROBEXT_DOCUMENT_CLASS__}{standalone}
\setPlaceholder{__ROBEXT_PREAMBLE__}{}
\setPlaceholder{__ROBEXT_PREAMBLE_HYPERREF__}{}
\setPlaceholder{__ROBEXT_PREAMBLE_AFTER_HYPERREF__}{}
\setPlaceholder{__ROBEXT_INCLUDEGRAPHICS_OPTIONS__}{}
\setPlaceholder{__ROBEXT_INCLUDEGRAPHICS_FILE__}{\robExtAddCachePathAndName{\robExtFinalHash.pdf}}
\setPlaceholder{__ROBEXT_LATEX_TRIM_LENGTH__}{30cm}

\begin{PlaceholderFromCode}{__ROBEXT_LATEX__}
\documentclass[__ROBEXT_LATEX_OPTIONS__]{__ROBEXT_DOCUMENT_CLASS__}
__ROBEXT_PREAMBLE__
% most packages must be loaded before hyperref
% so we typically want to load hyperref here
__ROBEXT_PREAMBLE_HYPERREF__
% some packages must be loaded after hyperref
__ROBEXT_PREAMBLE_AFTER_HYPERREF__
\begin{document}%
__ROBEXT_MAIN_CONTENT_WRAPPED__
\end{document}
\end{PlaceholderFromCode}

\begin{PlaceholderFromCode}{__ROBEXT_MAIN_CONTENT_WRAPPED__}
__ROBEXT_CREATE_OUT_FILE__%
\newsavebox\boxRobExt%
\begin{lrbox}{\boxRobExt}%
  __ROBEXT_MAIN_CONTENT__%
\end{lrbox}%
\usebox{\boxRobExt}%
__ROBEXT_WRITE_DEPTH_TO_OUT_FILE__%
\end{PlaceholderFromCode}

\begin{PlaceholderFromCode}{__ROBEXT_CREATE_OUT_FILE__}
%% We save the height/depth of the content by using a savebox:
\newwrite\writeRobExt%
\immediate\openout\writeRobExt=\jobname-out.tex%
\end{PlaceholderFromCode}

\begin{PlaceholderFromCode}{__ROBEXT_WRITE_DEPTH_TO_OUT_FILE__}
\immediate\write\writeRobExt{%
  \string\def\string\robExtWidth{\the\wd\boxRobExt}%
  \string\def\string\robExtHeight{\the\ht\boxRobExt}%
  \string\def\string\robExtDepth{\the\dp\boxRobExt}%
}%
\end{PlaceholderFromCode}

%% Compilation commands
\setPlaceholder{__ROBEXT_COMPILATION_COMMAND_LATEX__}{__ROBEXT_LATEX_ENGINE__ __ROBEXT_COMPILATION_COMMAND_OPTIONS__ "__ROBEXT_SOURCE_FILE__"}
\setPlaceholder{__ROBEXT_COMPILATION_COMMAND_OPTIONS__}{-halt-on-error}
\setPlaceholder{__ROBEXT_LATEX_ENGINE__}{pdflatex}

\robExtConfigure{
  % some useful presets
  latex/.style={
    set template={__ROBEXT_LATEX__},
    set compilation command={__ROBEXT_COMPILATION_COMMAND_LATEX__},
    %% Configure the latex compilation engine
    use latexmk/.style={
      set placeholder={__ROBEXT_LATEX_ENGINE__}{latexmk},
    },
    use lualatex/.style={
      set placeholder={__ROBEXT_LATEX_ENGINE__}{lualatex},
    },
    use xelatex/.style={
      set placeholder={__ROBEXT_LATEX_ENGINE__}{xelatex},
    },
    set latex options/.style={
      set placeholder={__ROBEXT_LATEX_OPTIONS__}{##1},
    },
    add to latex options/.style={
      add to placeholder no space={__ROBEXT_LATEX_OPTIONS__}{,##1},
    },
    set documentclass/.style={
      set placeholder={__ROBEXT_DOCUMENT_CLASS__}{##1},
    },
    set preamble/.style={
      set placeholder={__ROBEXT_PREAMBLE__}{##1},
    },
    add to preamble/.style={
      add to placeholder={__ROBEXT_PREAMBLE__}{##1},
    },
    add before preamble/.style={
      add before placeholder={__ROBEXT_PREAMBLE__}{##1},
    },
    set preamble hyperref/.style={
      set placeholder={__ROBEXT_PREAMBLE_HYPERREF__}{##1},
    },
    add to preamble hyperref/.style={
      add to placeholder={__ROBEXT_PREAMBLE_HYPERREF__}{##1},
    },
    set preamble after hyperref/.style={
      set placeholder={__ROBEXT_PREAMBLE_AFTER_HYPERREF__}{##1},
    },
    add to preamble after hyperref/.style={
      add to placeholder={__ROBEXT_PREAMBLE_AFTER_HYPERREF__}{##1},
    },
    do not wrap code/.style={
      set placeholder={__ROBEXT_MAIN_CONTENT_WRAPPED__}{__ROBEXT_MAIN_CONTENT__},
    },    
    add to includegraphics options={trim=__ROBEXT_LATEX_TRIM_LENGTH__ __ROBEXT_LATEX_TRIM_LENGTH__ __ROBEXT_LATEX_TRIM_LENGTH__ __ROBEXT_LATEX_TRIM_LENGTH__},
    add to latex options={margin=__ROBEXT_LATEX_TRIM_LENGTH__},
    do not add margins/.style={
      set placeholder={__ROBEXT_LATEX_TRIM_LENGTH__}{0cm}
    },
  },
  % U
  tikz/.style={
    latex,
    add to preamble={\usepackage{tikz}},
  },
  tikzpicture/.style={
    tikz,
    set placeholder={__ROBEXT_MAIN_CONTENT__}{\begin{tikzpicture}__ROBEXT_MAIN_CONTENT_ORIG__\end{tikzpicture}},
  },
}

% %%%%%%% Integration with python


\begin{PlaceholderFromCode}{__ROBEXT_PYTHON__}
__ROBEXT_PYTHON_IMPORT__
__ROBEXT_PYTHON_MAIN_CONTENT_WRAPPED__
\end{PlaceholderFromCode}

\setPlaceholder{__ROBEXT_PYTHON_IMPORT__}{}

\begin{PlaceholderFromCode}{__ROBEXT_PYTHON_MAIN_CONTENT_WRAPPED__}
# This file will be loaded in latex. Useful to pass data to the main document
f_out_write = open("__ROBEXT_OUTPUT_PREFIX__-out.tex", "w")

import os
import sys

def write_to_out(text):
    """Write to the -out.tex file that is loaded by default"""
    f_out_write.write(text)

def parse_args():
    args = {}
    if len(sys.argv) % 2 == 0:
        print("Error: the number of arguments must be even, as tuples of name and value")
        exit(1)
    for i in range(0,len(sys.argv)-1,2):
        args[sys.argv[i+1]] = sys.argv[i+2]
    return args

def get_cache_folder():
    '''
    Path of the cache folder. Warning: this works only when the python script
    is located in this cache folder (that should be true when it's called from LaTeX)
    '''
    return os.path.abspath(os.path.dirname(sys.argv[0])) 

def get_file_base():
    '''
    Outputs the base of the files (i.e. something like robExt-somehash, without any extension)
    '''
    return os.path.splitext(os.path.basename(sys.argv[0]))[0] # __file__ does not work as it refers to the library

def get_current_script():
    '''
    Outputs the path of the current script
    '''
    return os.path.abspath(sys.argv[0]) # __file__ does not work as it refers to the library

    
def get_filename_from_extension(extension):
    '''
    If you want to create a file with extension 'extension' (with the appropriate base name), this command
    is for you. For instance get_filename_from_extension(".mp4") would return something like
    robExt-somehash.mp4
    the extension can also be like get_filename_from_extension("-out.tex") etc.
    '''
    return os.path.join(get_cache_folder(), get_file_base() + extension)

def get_verbatim_output():
    '''Returns the path to -out.txt that is read by verbatim output'''    
    return get_filename_from_extension("-out.txt")

def get_pdf_output():
    '''Returns the path to -out.txt that is read by verbatim output'''    
    return get_filename_from_extension(".pdf")

    
def finished_with_no_error():
    '''
    Call this at the end of your script. This creates the path of the final pdf file that should be
    created (otherwise robust-externalize will think that the compilation failed)
    '''
    if not os.path.exists(get_filename_from_extension(".pdf")):
        # we create an empty path
        with open(get_filename_from_extension(".pdf"), 'w') as f:
            pass

### Starting main content
__ROBEXT_MAIN_CONTENT__
### Ending main content
__ROBEXT_PYTHON_FINISHED_WITH_NO_ERROR__
f_out_write.close()
\end{PlaceholderFromCode}

% It is annoying to manually call finished_with_no_error(), but it is handy to be able to disable it.
\begin{PlaceholderFromCode}{__ROBEXT_PYTHON_FINISHED_WITH_NO_ERROR__}
finished_with_no_error()
\end{PlaceholderFromCode}

%% On windows, python3 does not exist, and python points to python3. On linux, it seems to depend, at least on
%% my system it points to python3 as well.
\setPlaceholder{__ROBEXT_PYTHON_EXEC__}{python}

\robExtConfigure{
  python/.style={
    set compilation command={__ROBEXT_PYTHON_EXEC__ "__ROBEXT_SOURCE_FILE__"},
    set template={__ROBEXT_PYTHON__},
    print verbatim if no externalization,
    force python3/.style={
      set placeholder={__ROBEXT_PYTHON_EXEC__}{python3}
    },
  }
}

%% A style to print both the code and the result:
\begin{PlaceholderFromCode}{__ROBEXT_PYTHON_PRINT_CODE_RESULT_TEMPLATE_BEFORE__}
# File where print("bla") should be redirected
# get_filename_from_extension("-foo.txt") will give you the path of the file
# in the cache that looks like robExt-somehash-foo.txt
print_file = open(get_filename_from_extension("-print.txt"),  "w")
sys.stdout = print_file
# This code will read the current code, and extract the lines between
# that starts with "### CODESTARTSHERE" and "### CODESTOPSHERE", and will write
# it into the *-code.text (we do not want to print all these functions in
# the final code)
with open(get_filename_from_extension("-code.txt"), "w") as f:
    # The current script has extension .tex
    with open(get_current_script(), "r") as script:
        should_write = False
        for line in script:
            if line.startswith("### CODESTARTSHERE"):
                should_write = True
            elif line.startswith("### CODESTOPSHERE"):
                should_write = False
            elif "HIDEME" in line:
                pass
            else:
                if should_write:
                    f.write(line)
### CODESTARTSHERE
\end{PlaceholderFromCode}


\begin{PlaceholderFromCode}{__ROBEXT_PYTHON_PRINT_CODE_RESULT_TEMPLATE_AFTER__}
### CODESTOPSHERE
print_file.close()
\end{PlaceholderFromCode}
\setPlaceholder{__ROBEXT_PYTHON_TCOLORBOX_PROPS__}{colback=red!5!white,colframe=red!75!black}
\setPlaceholder{__ROBEXT_PYTHON_CODE_MESSAGE__}{}
\setPlaceholder{__ROBEXT_PYTHON_RESULT_MESSAGE__}{Output:}
\setPlaceholder{__ROBEXT_PYTHON_LSTINPUT_STYLE__}{frame=single, breakindent=.5\textwidth, frame=single, breaklines=true, style=mypython}
\robExtConfigure{
  python print code and result/.style={
    python,
    add before placeholder no space={__ROBEXT_MAIN_CONTENT__}{__ROBEXT_PYTHON_PRINT_CODE_RESULT_TEMPLATE_BEFORE__},
    add to placeholder no space={__ROBEXT_MAIN_CONTENT__}{__ROBEXT_PYTHON_PRINT_CODE_RESULT_TEMPLATE_AFTER__},
    set title/.style={
      set placeholder={__MY_TITLE__}{##1},
    },
    set title={Python code},
    custom include command={
      % Useful to replace __MY_TITLE__:
      \evalPlaceholder{
        \begin{tcolorbox}[title=__MY_TITLE__,__ROBEXT_PYTHON_TCOLORBOX_PROPS__]
          __ROBEXT_PYTHON_CODE_MESSAGE__%
          \lstinputlisting[__ROBEXT_PYTHON_LSTINPUT_STYLE__]{\robExtAddCachePathAndName{\robExtFinalHash-code.txt}}
          __ROBEXT_PYTHON_RESULT_MESSAGE__%
          \verbatiminput{\robExtAddCachePathAndName{\robExtFinalHash-print.txt}}
        \end{tcolorbox}
      }
    },
  },
}


%%%% Verbatim text

\robExtConfigure{
  verbatim text/.style={
    set template={__ROBEXT_MAIN_CONTENT__},
    custom include command={\evalPlaceholder{\verbatiminput{\robExtAddCachePathAndName{\robExtFinalHash.tex}}}},
    %% Apparently this works on windows as well https://stackoverflow.com/questions/1702762/how-can-i-create-an-empty-file-at-the-command-line-in-windows
    set compilation command={echo "" > __ROBEXT_OUTPUT_PDF__},
  },
  verbatim text no include/.style={
    verbatim text,
    custom include command={\evalPlaceholder{%
        \xdef\robExtPathToInput{\robExtAddCachePathAndName{\robExtFinalHash.tex}}%
      }%
    }%
  },
}

%%%%% Bash

\begin{PlaceholderFromCode}{__ROBEXT_BASH_TEMPLATE__}
# Quit if there is an error
set -e
outputTxt="__ROBEXT_OUTPUT_PREFIX__-out.txt"
outputTex="__ROBEXT_OUTPUT_PREFIX__-out.tex"
outputPdf="__ROBEXT_OUTPUT_PDF__"
__ROBEXT_MAIN_CONTENT__
# Create the pdf file to certify that no compilation error occured
touch "${outputPdf}"
\end{PlaceholderFromCode}

\setPlaceholder{__ROBEXT_BASH_SHELL__}{bash}

\robExtConfigure{
  bash/.style={
    set compilation command={__ROBEXT_BASH_SHELL__ "__ROBEXT_SOURCE_FILE__"},
    set template={__ROBEXT_BASH_TEMPLATE__},
    print verbatim if no externalization,
  }
}


%%%%% Replace tikzpicture:

\NewDocumentCommand{\robExtExternalizeAllTikzpictures}{O{<>}}{%
  \robExtCacheEnvironment[#1]{tikzpicture}{tikzpicture}
  %\robExtCacheCommand[#1]{tikz}[O{}m]{tikz}
}


%% The cached version
\DeclareDocumentEnvironment{tikzpictureC}{D<>{}O{}O{}}{%
  \begin{CacheMe}{tikzpicture,#1}[#2]%
}{\end{CacheMe}}%

%% \robExtCacheEnvironment{myenv}
\NewDocumentCommand{\robExtCacheEnvironment}{O{<>}mm}{%
  %% We need to save the original environment to avoid infinite recursion if we disable externalization
  %% https://tex.stackexchange.com/questions/695391/why-isnt-my-environment-restored/695398
  \NewEnvironmentCopy{robExtEnvironmentOrig#2}{#2}%
  \robExtAddToEnvironmentResetList{#2}%
  \DeclareDocumentEnvironment{#2}{D#1{}}{%
    \def\robExtEnvironmentOrigName{#2}%
    \CacheMe{%
      #3,%
      set placeholder={__ROBEXT_MAIN_CONTENT__}{\begin{#2}__ROBEXT_MAIN_CONTENT_ORIG__\end{#2}},
      ##1%
    }%
  }{\endCacheMe}%
}
\let\cacheEnvironment\robExtCacheEnvironment


%%%% Borrowed and adapted from https://github.com/sasozivanovic/memoize/blob/master/xparse-arglist.sty
%%%% see also https://tex.stackexchange.com/questions/695662/automatically-wrap-a-macro/695734
%% The idea of the library is that it builds a string like
%% [#2]<#3>{#4}
%% in order to generate something like
%% \NewDocumentCommand{\myfunction}{D<>{}O{coucou}D<>{yes}m}
%% {
%%  \cacheMe[#1]{\myfunction[#2]<#3>{#4}}
%% }
%%%%   _________________________________________________________________

\def\robExtArgumentList{%
  \expandafter\robExt@arglist\expandafter0\ArgumentSpecification.%
}

\def\robExt@arglist#1#2{%
  \ifcsname robExt@arglist@#2\endcsname
    \csname robExt@arglist@#2\expandafter\expandafter\expandafter\endcsname
  \else
    \expandafter\robExt@arglist@error
  \fi
  \expandafter{\the\numexpr#1+1\relax}%
}

% \robExt@arglist@...: #1 = the argument number
\def\robExt@arglist@m#1{\noexpand\unexpanded{{#####1}}\robExt@arglist{#1}}
\def\robExt@arglist@r#1#2#3{\noexpand\unexpanded{#2#####1#3}\robExt@arglist{#1}}
\def\robExt@arglist@R#1#2#3#4{\noexpand\unexpanded{#2#####1#3}\robExt@arglist{#1}}
\def\robExt@arglist@v#1{{Handled commands with verbatim arguments are not
    supported}\robExt@arglist{#1}} % error
\def\robExt@arglist@b#1{{This is not the way to handle
    environment}\robExt@arglist{#1}} % error
\def\robExt@arglist@o#1{\noexpand\unexpanded{[#####1]}\robExt@arglist{#1}}
\def\robExt@arglist@d#1#2#3{\noexpand\unexpanded{#2#####1#3}\robExt@arglist{#1}}
\def\robExt@arglist@O#1#2{\noexpand\unexpanded{[#####1]}\robExt@arglist{#1}}
% \def\robExt@arglist@O#1#2{\noexpand\unexpanded{[#####1]}\robExt@arglist{#1}}
\def\robExt@arglist@D#1#2#3#4{\noexpand\unexpanded{#2#####1#3}\robExt@arglist{#1}}
\def\robExt@arglist@s#1{\noexpand\IfBooleanT{#####1}{*}\robExt@arglist{#1}}
\def\robExt@arglist@t#1#2{\noexpand\IfBooleanT{#####1}{#2}\robExt@arglist{#1}}
\csdef{robExt@arglist@+}#1{\expandafter\robExt@arglist\expandafter{\the\numexpr#1-1\relax}}%
\csdef{robExt@arglist@.}#1{}
% e,E: Embellishments are not supported.
% > Argument processors are not supported. And how could they be?
\def\robExt@arglist@error#1.{{Unknown argument type}}
%%%% _________________________________________________________________

%% Since the very first occurrence of D is not forwarded to the function
%% we discard it:
%% The first argument of \robExt@arglist@D is the number of the argument
%% so #2#####1#3 reads as #2 = <, #### = #, #1 = > #3 = default value that can be discarded since it is already part of the argument spec.

\def\robExt@arglist@D#1#2#3#4{%
  \noexpand\unexpanded{%
    \ifnum #1=1 % If it is the first argument D, then we do not add anything to the argument string
    \else #2#####1#3\fi}\robExt@arglist{#1}%
}

\ExplSyntaxOn
\NewDocumentCommand{\robExtRescanHashRobust}{m}{
  \str_set_hash_robust:Nn \l_robExt_tmp_str {#1}
  \tl_rescan:nv {}{ l_robExt_tmp_str }
}
\ExplSyntaxOff

%% Inspired and modified from
%% https://github.com/sasozivanovic/memoize/blob/81d960aa547148bdb38fea89917eda1476c9bace/memoize.sty#L744
\NewDocumentCommand{\robExtCacheCommand}{O{<>}mom}{%
  %% We get the specification of the command, like "O{}mm"
  \IfNoValueTF{#3}{%
    \expandafter\GetDocumentCommandArgSpec\csname #2\endcsname%
  }{%
    \def\ArgumentSpecification{#3}%
  }%
  \edef\ArgumentSpecification{D#1{}\ArgumentSpecification}%
  % We copy the original definition for later (if externalization is disabled)
  \expandafter\DeclareCommandCopy\csname robExtCommandOrig#2\expandafter\endcsname\csname #2\endcsname%
  \robExtAddToCommandResetList{#2}%
  \edef\robExt@marshal{%
    \noexpand\DeclareDocumentCommand%
      \expandonce{\csname #2\endcsname}%
      {\expandonce{\ArgumentSpecification}}%
      {%
        \noexpand\def\noexpand\robExtCommandOrigName{#2}%
        % todo: add a hook for users setup; prevent user from changing \MemoizeWrapper?
        \edef\noexpand\robExt@marshal{%
        \noexpand\noexpand\noexpand\robExtRescanHashRobust{\noexpand\noexpand\noexpand\robExtCacheMe[\detokenize{#4}, \noexpand\detokenize{####1}]{%
          \noexpand\noexpand\expandonce{\csname #2\endcsname}%
            \robExtArgumentList%
          }%
        }%
      }%
      % For debug
      %\noexpand\show\noexpand\robExt@marshal%
      \noexpand\robExt@marshal%
    }%
  }%
  % for debug
  %\show\robExt@marshal%
  \robExt@marshal%
}
\let\cacheCommand\robExtCacheCommand



\makeatother
% Loads the great package that produces tikz-like manual (see also tikzcd for examples)
\input{pgfmanual-en-macros.tex} % Is supposed to be included in recent TeX distributions, but I get errors...
\usepackage{makeidx} % Produces an index of commands.
\makeindex % Useful or not index will be created
\usepackage{alertmessage} % For warning, info...
\newcommand{\mylink}[2]{\href{#1}{#2}\footnote{\url{#1}}}
\usepackage{verbatim}
\usepackage{mathtools}
\usepackage{float} %figure inside minipage
\usepackage{listings}
\usepackage[hidelinks]{hyperref}
\usepackage{cleveref}


\begin{document}
%%% Title: thanks tikzcd for the styling
\begin{center}
  \vspace*{1em} % Thanks tikzcd
  \tikz\node[scale=1.2]{%
    \color{gray}\Huge\ttfamily \char`\{\raisebox{.09em}{\textcolor{red!75!black}{robust\raisebox{-0.1em}{-}externalize}}\char`\}};

  \vspace{0.5em}
  {\Large\bfseries Cache anything (\tikzname, python…),\\in a robust, efficient and pure way.}

  \vspace{1em}
  {Léo Colisson \quad Version 2023/03/22-unstable}\\[3mm]
  {\href{https://github.com/leo-colisson/robust-externalize}{\texttt{github.com/leo-colisson/robust-externalize}}}
\end{center}

\tableofcontents

\bigskip

\textbf{WARNING: This library is very young and has not been tested extensively. Even if we try to stay backward compatible, the only guaranteed way to be immune to changes is to copy/paste the library in your main project folder.}

\section{Introduction}

\subsection{Why do I need to cache (a.k.a. externalize) parts of my document?}

One often wants to cache (i.e.\ store pre-compiled parts of the document, like figures) operations that are long to do: For instance, TikZ is great, but TikZ figures often takes time to compile (it can easily take a few seconds per picture). This can become really annoying with documents containing many pictures, as the compilation can take multiple minutes: for instance my thesis needed roughly 30mn to compile as it contains many tiny figures, and LaTeX needs to compile the document multiple times before converging to the final result. But even on much smaller documents you can easily reach a few minutes of compilation, which is not only high to get a useful feedback in real time, but worse, when using online \LaTeX{} providers (e.g. overleaf), this can be a real pain as you are unable to process your document due to timeouts.

Similarly, you might want to cache the result of some codes, for instance a text or an image generated via python and matplotlib, without manually compiling them externally.

\subsection{Why not using \tikzname{}'s externalize library?}

\tikzname{} has an externalize library to pre-compile these images on the first run. Even if this library is quite simple to use, it has multiple issues:
\begin{itemize}
\item If you add a picture before existing pre-compiled pictures, the pictures that are placed after will be recompiled from scratch. This can be mitigated by manually adding a different prefix to each picture, but it is highly not practical to use.
\item To compile each picture, TikZ's externalize library reads the document's preambule and needs to process (quickly) the whole document. In large documents (or in documents relying on many packages), this can result in a significant loading time, sometimes much bigger than the time to compile the document without the externalize library: for instance, if the document takes 10 seconds to be processed, and if you have 200 pictures that take 1s each to be compiled, the first compilation with the TikZ's externalize library will take roughly half an hour instead of 3mn without the library. And if you add a single picture at the beginning of the document… you need to restart everything from scratch. For these reasons, I was not even able to compile my thesis with TikZ's external library in a reasonable time.
\item  If two pictures share the same code, it will be compiled twice
\item Little purity is enforced: if a macro changes before a pre-compiled picture that uses this macro, the figure will not be updated. This can result in different documents depending on whether the cache is cleared or not.
\item As far as I know, it is made for TikZ picture mostly, and is not really made for inserting other stuff, like matplotlib images generated from python etc...
\item According to some maintainers of TikZ, ``\mylink{https://github.com/pgf-tikz/pgf/issues/758}{the code of the externalization library is mostly unreadable gibberish}'', and therefore most of the above issues are unlikely to be solved in a foreseable future.
\end{itemize}

\subsection{FAQ}

\paragraph{Do you need to compile using -shell-escape?}

Since we need to compile the images via an external command, the simpler option is to add the argument |-shell-escape| to let the library run the  compilation command automatically (this is also the case of \tikzname's externalize library). However, people worried by security issues of |-shell-escape| (that allows arbitrary code execution if you don't trust the \LaTeX{} code) might be interested by these facts:
\begin{itemize}
\item If images are all already cached, you don't need to enable \texttt{-shell-escape}.
\item You can choose not to compile non-cached content, and display a dummy content instead until you choose to compile them.
\item You can compile manually the images: all the commands that are left to be run are listed in \texttt{robExt-compile-missing-figures.sh} and you can just run them, either with \texttt{bash robExt-compile-missing-figures.sh} or by typing them manually (most of the time it's only a matter of running \texttt{pdflatex somefile.tex}). 
\end{itemize}


\paragraph{Is it working on overleaf?}

Yes: overleaf automatically compiles documents with |-shell-escape|, so nothing special needs to be done there (of course, if you use this library to run some code, the programming language might not be available, but I heard that python is installed on overleaf servers for instance, even if this needs to be doubled checked). If the first compilation of the document to cache images times out, you can just repeat this operation multiple times until all images are cached.

\paragraph{Do you have some benchmarks?}

On an early draft of a small paper containing 76 small tikz-cd based pictures (from my other zx-calculus library), we measured:
\begin{itemize}
\item 35 seconds for a normal compilation without externalization
\item 75 seconds for the first compilation with this library
\item 2.4 seconds for the next runs
\end{itemize}
So during the first compilation, we lost a x2 factor (roughly an additional time of .5 seconds per picture coming from the time to start \LaTeX{}, it seems like on average a picture takes .5 seconds to be built in my benchmark), but then we have a speedup of x15 (2.43s instead of 34.63s) for all subsequent runs. And I expect this to be even higher with more pictures and more complex documents.

\paragraph{Can you deal with baseline position?} Yes, the depth of the box is automatically computed and used to include the figure by default.

\paragraph{How is purity enforced?} Purity is the property that if you remove the cached files and recompile your document, you should end-up with the same output. To enforce purity, we compute the hash of the final program, including the compilation command and the dependency files used for instance in |\input{include.tex}| (unless you prefer not to, for instance to keep parts of the process impure for efficiency reasons), and put the code in a file named based on this hash. Then we compile it if it has not been used before, and include the output. Changing a single character in the file, the tracked dependencies, or the compilation command will lead to a new hash, and therefore to a new generated picture.

\paragraph{Can I extend it easily?} We tried to take a quite modular approach in order to allow easy extensions. Internally, to support a new cache scheme, we only expect a string containing the program (possibly produced using a template), a list of dependencies, a command to compile this program (e.g. producing a pdf and possibly a tex file with the properties (depth…) of the pdf), and a command to load the result of the compilation into the final document (called after loading the previously mentioned optional tex file). Thanks to pgfkeys, it is then possible to create simple pre-made settings to automatically apply when needed.

\section{Quickstart}

\subsection{Installation}

To install the library, just copy the |robust-externalize.sty| file into the root of the project. Then, load the library using:\\

|\usepackage{robust-externalize}|

\subsection{Usage}

\subsubsection{For \LaTeX{} based content}

In theory, if you only care about \tikzname's picture, you can just do:

\begin{codeexample}[width=0pt]
%% We override the default tikzpicture environment
%% to externalize all pictures  
\robExtExternalizeAllTikzpictures

\begin{tikzpicture}[baseline,anchor=base]
  \node[draw,rounded corners,fill=pink!60]{Hello World!};
\end{tikzpicture}
\end{codeexample}

\noindent and all tikzpictures created using |\begin{tikzpicture}...| will be ``externalized''. However, by default the \LaTeX{} template used to compile these pictures is empty, so you will quickly want to populate the template (for instance to load packages, define custom macros possibly shared with the main document etc.). So let's step back and see how we can define an arbitrary template.

If you want to compile a \LaTeX{} code, say a \tikzname{} picture, you first need to define the \LaTeX{} template that will wrap all your code\footnote{While you could use the same preambule as the main project, for instance using a shared \texttt{\textbackslash input\{input.tex\}} file, this is not recommended as it will not only be longer to load (some packages are useless to build your average tikzpictures), but it also harms purity or efficiency: if you choose to track this common input (i.e.\ dependency), then it will recompile the pictures every time you change \texttt{input.tex}, and if you don't track this dependency, then you might skip a needed recompilation leading to a different outcome after invalidating the cache.}. Because you might want to mix different templates in a document (e.g.\ for tikz pictures, matplotlib python code, tikz-cd or zx diagrams…), we like to define them into a preset that is just a set of configuration options\footnote{Internally this is just a pgfkeys style, if you don't want to define presets, you can just write the configuration outside of the preset… but this is not recommended, except for some specific configuration options like \texttt{disable externalization} that you way want to apply globally.}). For instance to create a basic preset called |presetTikz|, use:

\begin{codeexample}[width=0pt]
  \robExtConfigure{%
    presetTikz/.style={
      % We define the code that wraps all our figures
      defineTemplate={
        \documentclass{standalone} % standalone ensures that the pdf output size matches the content
        \usepackage{tikz} % Loads any package you need to compile your pictures
        % You can define macros, just make sure to double the number of sharps
        % as otherwise they will be understood as options of the preset.
        \def\sayHello##1{Hello ##1}
        \input{input_externalize.tex} % you can put in this file regular LaTeX code to share with the main document
        \begin{document}%
        \robExtMainContent% This macro will be replaced with (notably) the code for the figure
        \end{document}
      },
      % The dependency files needed to compile the file
      % (included when computing the hash). Separate them with commas.
      dependencies={input_externalize.tex},
    },
  }
\end{codeexample}
% For the example to work:
\robExtConfigure{%
  presetTikz/.style={
    % We define the code that wraps all our figures
    defineTemplate={
      \documentclass{standalone} % standalone ensures that the pdf output size matches the content
      \usepackage{tikz} % Loads any package you need to compile your pictures
      % You can define macros, just make sure to double the number of sharps
      % as otherwise they will be understood as options of the preset.
      \def\sayHello##1{Hello ##1}
      \begin{document}%
      \robExtMainContent% This macro will be replaced with (notably) the code for the figure
      \end{document}
    },
    % The dependency files needed to compile the file
    % (included when computing the hash). Separate them with commas.
    dependencies={input_externalize.tex},
  },
}

(see the comments for details)

Then, if needed, create the dependency files, and use this preset as follows in your code:

\begin{codeexample}[width=0pt]
See that the baseline is respected %
\begin{robExtern}{presetTikz}%
  \begin{tikzpicture}[baseline,anchor=base]%
    \node[draw,rounded corners,fill=pink!60]{\sayHello{World}!};
  \end{tikzpicture}
\end{robExtern}
\end{codeexample}

Note that one might be tempted to move |\begin{tikzpicture}| inside the template to save a bit of typing. However, our library is actually replacing |\robExtMainContent| with the content wrapped around some code\footnote{Basically creating a box in order to compute the depth of the content, and write it to a tex file to use for later.} that need to wrap the whole content. While we provide other macros that don't add this wrapping code\footnote{But then we need to manually add this box if we want to compute the appropriate depth, and we will need to redefine the command to disable externalization.}, it is actually simpler, more configurable, and more typing-friendly, to define an environment that automatically picks the right preset and adds |\begin{tikzpicture}| automatically for us:

\begin{codeexample}[width=0pt]
  \DeclareDocumentEnvironment{mytikzpicture}{O{}O{}b}{% = 2 optional arguments + the body (b), cf xparse
    \begin{robExtern}{presetTikz,#2}%
      \begin{tikzpicture}[#1]%
        #3
      \end{tikzpicture}%
    \end{robExtern}%
  }{}
\end{codeexample}

This way, in your code you can just use:
{
  \DeclareDocumentEnvironment{mytikzpicture}{O{}O{}b}{%
    \begin{robExtern}{presetTikz,#2}%
      \begin{tikzpicture}[#1]%
        #3%
      \end{tikzpicture}%
    \end{robExtern}%
  }{}
\begin{codeexample}[width=0pt]
See the respected baseline: %
\begin{mytikzpicture}[baseline,anchor=base]
  \node[draw,rounded corners,fill=pink!60]{\sayHello{World}!};
\end{mytikzpicture}
\end{codeexample}
}
    
By choosing the name |tikzpicture| instead of |mytikzpicture|, you would actually override \tikzname's macro, which should be perfectly fine if you want to externalize \tikzname's pictures by default. Because this usecase will likely be important, we actually provide a command that defines a similar environment, except that it uses a preset called |presetTikzDefault| (this style is populated with a very simple template, but you surely want to quickly override it with your own template by just creating a new preset called |presetTikzDefault|):

\begin{codeexample}[width=0pt]
%% We override the default tikzpicture environment:
\robExtExternalizeAllTikzpictures

\begin{tikzpicture}[baseline,anchor=base]
  \node[draw,rounded corners,fill=pink!60]{Hello World!};
\end{tikzpicture}
\end{codeexample}



\subsection{For non-\LaTeX{} code}

Due to the way \LaTeX{} works, non-\LaTeX{} code can't be reliably read inside macros and some environments (e.g. align) as some characters are removed (e.g. percent symbol). For this reason, we sometimes need to separate the time where we define the code and where we insert it, and we provide therefore different commands to deal with non-\LaTeX{} code.

The following code will name a template |pythonMatplotlib| (see how we use |ROBEXTMAINCONTENT| as a placeholder for the content), define a preset based on this template, and :

\begin{codeexample}[width=0pt]
% We define our python template:
\begin{robExtNamedTemplate}[pythonMatplotlib]
import matplotlib.pyplot as plt
import sys
ROBEXTMAINCONTENT
plt.savefig(sys.argv[1]+".pdf")
\end{robExtNamedTemplate}

% We define a reusable preset, note that we specify the compilation command:
\robExtConfigure{
  presetMatplot/.style={
    defineTemplateFromName=pythonMatplotlib,
    set compilation command={python3 "\robExtFinalFile" "\robExtFinalPrefixedName"},
  },
}

%%%%%%%%%%%%%%% Above code must be written once, below is used for any drawing

% We draw our code:
\begin{robExtCode}{presetMatplot,include graphics args={width=.5\linewidth}}
year = [2014, 2015, 2016, 2017, 2018, 2019]  
tutorial_count = [39, 117, 111, 110, 67, 29]
plt.plot(year, tutorial_count, color="#6c3376", linewidth=3)  
plt.xlabel('Year')  
plt.ylabel('Number of futurestud.io Tutorials') 
\end{robExtCode}
\end{codeexample}

Note that for the fundamental reasons mentioned above, the above code can't work inside a macro. If we still want to include the result in a macro, we can separate the definition of the code and its usage:
\begin{codeexample}[width=0pt]
% We define our python template:
\begin{robExtNamedTemplate}[pythonMatplotlib]
import matplotlib.pyplot as plt
import sys
ROBEXTMAINCONTENT
plt.savefig(sys.argv[1]+".pdf")
\end{robExtNamedTemplate}

% We define a reusable preset, note that we specify the compilation command:
\robExtConfigure{
  presetMatplot/.style={
    defineTemplateFromName=pythonMatplotlib,
    set compilation command={python3 "\robExtFinalFile" "\robExtFinalPrefixedName"},
  },
}

%%%%%%%%%%%%%%% Above code must be written once, below is used for any drawing

% We define our code (implicitely giving it a default name, that can be changed
% if multiple codes are inserted in the same macro)
\begin{robExtNamedContent}
year = [2014, 2015, 2016, 2017, 2018, 2019]  
tutorial_count = [39, 117, 111, 110, 67, 29]

plt.plot(year, tutorial_count, color="#6c3376", linewidth=3)  
plt.xlabel('Year')  
plt.ylabel('Number of futurestud.io Tutorials') 
\end{robExtNamedContent}

Inside a macro: %
\fbox{\robExternPrev{presetMatplot,include graphics args={width=.7\linewidth}}}
\end{codeexample}

\subsection{Example of more advanced setup}

We can actually do many more things. For instance, here we define another preset that compiles the document with python and matplotlib, and displays the important lines of the code above the result, inside a figure with a customizable caption:

\begin{codeexample}[vbox]
% Define the template. Lines with TEMPLATECODE
% will be removed later.  
\begin{robExtNamedTemplate}[pythonMatplotlib]
import matplotlib.pyplot as plt # TEMPLATECODE
import sys # TEMPLATECODE
ROBEXTMAINCONTENT
plt.savefig(sys.argv[1]+".pdf") # TEMPLATECODE
\end{robExtNamedTemplate}

\robExtConfigure{
  % More complex version that displays both the code and the result:
  presetMatplotAdvanced/.style={
    /robExt/caption/.code={\gdef\mycaption{##1}}, % Provide a key to change the caption
    defineTemplateFromName=pythonMatplotlib,
    % This command compiles the image, and creates a file removing all the lines from the code
    % containing TEMPLATECODE (useful not to display action of saving the file etc)
    % This might not be portable to windows without installing cygwin, but one can replace sed
    % with some python code doing the same for a more portable code.
    set compilation command={python3 "\robExtFinalFile" "\robExtFinalPrefixedName" %
      && sed '/TEMPLATECODE/d' "\robExtFinalFile" > "\robExtFinalPrefixedName.codeonly.py"},
    custom include command={%
      \begin{figure}[H] % Use H mostly to avoid compilation error in documentation
        \centering
        % Note that this will display the template around the code. We could avoid this by putting in our
        % template a part that creates a new file whose name is the basename of the current (script) file
        % and that contains all its code except for the template (to differentiate between template and
        % non-template, we could add a special comment on lines to remove)
        \verbatiminput{\robExtAddPrefixPathAndName{\robExtFinalName.codeonly.py}}
        \includegraphics[width=.6\textwidth]{\robExtAddPrefixPathAndName{\robExtFinalName.pdf}}%
        \caption{\mycaption}
      \end{figure}
    },
  },
}

%%%%%%%%%%%%%%% Above code must be written once, below is used for any drawing

\begin{robExtCode}{presetMatplotAdvanced, caption={Here is my caption for the figure}}
year = [2014, 2015, 2016, 2017, 2018, 2019]  
tutorial_count = [39, 117, 111, 110, 67, 29]

plt.plot(year, tutorial_count, color="#6c3376", linewidth=3)  
plt.xlabel('Year')  
plt.ylabel('Number of futurestud.io Tutorials') 
\end{robExtCode}
\end{codeexample}

\section{Manual}

TODO: the example above already provide a nice view of the main functions, the |set subfolder={robustExternalize/}| option might be an important additionnal configuration option that allows you to put the cached pictures in a subfolder.

\section{Operations on the cache}

\subsection{Cleaning the cache}

You might want to clean the cache. Of course you can remove all generated files, but if you want to keep the picture in use in the latest version of the document, we provide a python script (automatically generated in the root folder) to do this. Just install python3 and run:\\

|python3 robExt-remove-old-figures.py|\\

You will then be prompted for a confirmation after providing the list of files that will be removed.

\subsection{Listing all figures in use}

After the compilation of the document, a file |robExt-all-figures.txt| is created with the list of the |.tex| file of all figures used in the current document.

\subsection{Manually compiling the figures}

When enabling the manual mode (useful if we don't want to enable |-shell-escape|):

|\robExtConfigure{|\\
|  enable manual mode|\\
|}|\\

the library creates a file |robExt-compile-missing-figures.sh| that contains the instructions to build the figures that are not yet in the cache. On Linux (or on Windows with bash/cygwin/… installed) you can easily execute them using:

|bash robExt-compile-missing-figures.sh|\\

\section{TODO and known bugs:}

\begin{itemize}
\item Solve problem with disable externalization not working with tikz pictures
\item We should create more pre-made settings, e.g. for tikz-cd, zx-calculus etc.
\item Some commands like mkdir might not be super compatible with Windows, I need to see how to improve compatibility between OS
\item The documentation is still sparse.
\item For now we put |compile-missing-figures.sh| in subfolders 
\end{itemize}
% %% This picture will NOT be externalized .
% \message{YYYYYYYYYYYYYY Next one:}
% \begin{tikzpicture}[baseline,anchor=base][disable externalization]
%   \node[draw,rounded corners,fill=pink!60]{Hello World!};
% \end{tikzpicture}


\end{document}
% Local Variables:
% TeX-command-extra-options: "-shell-escape -halt-on-error"
% End: